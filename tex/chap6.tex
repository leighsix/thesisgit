%# -*- coding: utf-8-unix -*-
% !TEX program = xelatex
% !TEX root = ../thesis.tex
% !TEX encoding = UTF-8 Unicode
%%==================================================
%% chapter02.tex for SJTU Master Thesis
%% based on CASthesis
%% modified by wei.jianwen@gmail.com
%% Encoding: UTF-8
%%==================================================

\chapter{Conclusion}
\label{chap6}
The features of competition on a two-layer network have been researched by changing network structures, switching updating rules, and selecting key nodes.\\
 
\section{Summary}
Many simulations have been carried out. In summary, it could be arranged as follows. 
In chapter~\ref{chap2}, the interconnected network with different dynamics on each layer is introduced to understand the competition on a two-layer network.  Also, many indexes are provided to measure and evaluate how a state of the network is changed. Based on this modeling, various simulations have been implemented according to 3 main topics as follows.

\begin{itemize}
\item Competition on a two-layer network with various structures
\item Competition with different updating rules
\item Influences of key nodes on competition
\end{itemize}

In chapter~\ref{chap3}, we investigated competition dynamics on two-layer networks with various structures. With changing network structures, it is measured and evaluated how the state of a network is changed and whether the network reaches consensus or not. As the method to revise structures of a network, three ways were provided, such as changing internal degrees, changing external degrees, and switching network types. First, as the result of changing internal degrees, it is found out that an internal degree on each layer has a different function. An internal degree on layer A tends to keep a positive state and to change a negative state into a positive state. Moreover, an internal degree on layer B tends to hinder a positive consensus state. Second, as the result of changing external degrees, \textit{Hierarchical Models} are provided. \textit{Hierarchical Models} show that an interconnected network is easy to make a consensus on both layers when the external degree in a decision-making layer is larger than a opinion layer, and the number of nodes in the decision-making layer is smaller than the opinion layer. Third, as the result of switching network types, there is no noticeable difference in the final state of a network. That means if there are no stubborn nodes, the influence of network types does not matter. However, it is shown that the number of internal edges has a more influential role for changing the state of a network than network types, and too many edges on each layer can cause that the networks are hard to reach a consensus due to inner conflicts.

In chapter~\ref{chap4}, it is researched how updating rules influence the competition of a two-layer network. Though updating rules are very various, we consider time-related updating rules, such as a simultaneous updating rule and a sequential updating rule. According to where the updating rules are applied, the simulations of three categories are implemented, such as layers' orders,  nodes' order, and links' order. Through simulation results, several conclusions are formed. First, a dynamics order between layers does not have a significant influence on changing the state of a network. Second, an order of edges in the layer A, that can be analyzed as characteristics of nodes such as `rash' and `considerate', has a vital influence on determining the final state of a network, such as the same orientation consensus, coexistence, and opposite orientation consensus. Third, an order of nodes in layer B, that can be analyzed as a communication method, is more influential for changing the state of a network than an order of nodes in layer A because it makes opinion convergence slow or fast. That means the communication method in the decision-making layer has a vital role in determining consensus time. Fourth, networks with simultaneous updating rules are easy to make slow consensus and coexistence or to change into the networks of the opposite state. Otherwise, networks with sequential updating rules are easy to make fast consensus states.

In chapter~\ref{chap5}, it is studied that how the key nodes can be selected on the various two-layer networks. To select key nodes on the various networks, we use single indicators and multiple indicators on various networks described in chapter~\ref{chap3}. Through the simulation results, several conclusions could be arranged as follows. First, the most effective method to identify key nodes is different according to network structures and layers, as shown in Table~\ref{effective methods}. Second, as single indicators, Pagerank, degree, and betweenness work well for selecting key nodes. Third, as multiple indicators, combined node centralities have good results to recognize the key nodes on various interconnected networks. 

Fourth, the larger number of links on layer A causes that key nodes make a consensus much quickly. Moreover, the larger number of links on layer B effects that key nodes make a consensus much harder. Fifth, as shown in the \textit{Hierarchical Model}, which is modeled by decreasing nodes on layer B and increasing an external degree on layer B, the \textit{Hierarchical Model} causes that the indexes for key nodes are hard to identify key nodes on layer B. Moreover, the \textit{Hierarchical Model} is easy to reach consensus by key nodes. Sixth, network types influence whether the network can make consensus by key nodes or not. Notably, it is observed that the \textit{RR} network is harder to reach consensus by key nodes and to recognize key nodes than the \textit{BA} network. \\
  
\section{Discussion} 
The competition of a two-layer network has been researched and analyzed under various conditions. It has been observed that how network structures influence the consensus of a two-layer network, how the updating rules affect the state of the network, what nodes have more influential for affecting the state of the network, and which method is a more effective way to identify critical nodes. Through these results, the state of a two-layer network might be controlled by managing the number of edges and the method of updating rules. Furthermore, for the best and fastest way to change the state of networks, the critical nodes might be recognized and controlled by using the method to select key nodes.
In the real world, we can find out the phenomenon of these competitions, such as election, legislation, adoption of new policies, and making-decision on social conflict issues. These competitions of the real world have similar characteristics with our simulation results. Therefore, based on simulation results, these competition models can be applied to solve social conflicts. As future work, it would be very attractive to make a generalized competition model with various structures and updating rules and to recognize key nodes on the generalized competition model.\\   

