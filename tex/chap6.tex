%# -*- coding: utf-8-unix -*-
% !TEX program = xelatex
% !TEX root = ../thesis.tex
% !TEX encoding = UTF-8 Unicode
%%==================================================
%% chapter02.tex for SJTU Master Thesis
%% based on CASthesis
%% modified by wei.jianwen@gmail.com
%% Encoding: UTF-8
%%==================================================

\chapter{Conclusion}
\label{chap:conclusion}
We have researched the competition of two layer networks. By changing network structures, switching updating rules and finding key nodes, the features of competition on two layers are found out. We hope that deficiency of this research would be researched forward and developed. 
\section{Summary}
So far, many simulations have been carried out. In summary, it can be arranged as follows. 
To begin with, competing interconnected networks with  different dynamics on each layer were introduced to understand the dynamics of competition on interconnected network.  And some indexes were provided to measure how the network state is changed and to evaluate the consensus on two layer. Based on this modeling, various simulations were implemented according to 3 main topics as follows.
\begin{itemize}
\item Competition on two layer with different structural network
\item Competition on two layer with different updating rules
\item Finding key nodes on two layer networks
\end{itemize}
In chapter 3, we have investigated competitions on two layer with various structural network. With changing network structures, it was measured and evaluated that how the interconnected network change its state and make consensus. As the method to revise the network structure, 3 ways were provided such as changing internal degrees, changing external degrees, and switching network types. 
First, as the result of changing the internal degrees, it could be found out that internal degrees on each layer has different features. The number of internal degrees on layer A has the tendency to keep positive state and to change negative state into positive state. And the number of internal degrees on layer B has the tendency to hinder positive state. Second, as the result of changing the external degrees, hierarchical models were provided. Hierarchical models show that it is easy to make consensus on both layers when the number of external edges in decision making is more than opinion layer. Third, as the result of switching the network type, there is no obvious result of network state change. If there are no stubborn nodes, network types do not matter, but internal degrees have more influential role for changing the state.\\

In chapter 4, it has been researched that how the updating rules have influence on the competition of two-layers network. Though updating rules are very various, we just have considered time-related updating rules, simultaneous rule or sequential rule. According to where the updating rules are applied, we have implemented the simulations of 3 categories, order of layers, order of nodes and order of links. Through simulation results, several conclusions could be arranged. First, dynamics order between layers does not have an influence for network state. Second, networks with more simultaneous updating rules are easy to make slow consensus or coexistence and to change into opposite state, otherwise networks with more sequential updating rules are easy to make fast consensus. It can be analyzed as that if opinion layer has more rash nodes, more time to have some conversation and decision making layer has more time to  discuss topics, the network have more probabilities to make consensus for opinion layer. Third, order of links in layer A is very influential so that it makes different network states. It can be explained as that characteristics of nodes in layer A, such as rash and considerate, affects the state of network. Forth, order of nodes in layer B are more influential for network states than order of nodes in layer A. It can be thought as that the communication method is very important in decision making layer. \\

In chapter 5, it has been studied that how the key nodes can be found out on the various interconnected network. To find key nodes on the network, we use single indicators and multiple indicators on various networks described in chapter 3. Through simulation results, several conclusions could be arranged.
First, the most effective method to identify key nodes is different according to network structures and layers as shown in Table.~\ref{effective methods}. Second, as single indicators, pagerank, degree and betweenness work well for finding key nodes. Second, as multiple indicators, combined node centrality totally has good results to recognize the key nodes on various networks. Third, decreasing the number of links on layer A makes it hard to find key nodes and to have consensus by stubborn nodes.  Fourth, decreasing the number of nodes on layer B makes it hard to identify key nodes and makes it easy to have consensus by stubborn nodes. Fifth, network types have the influence on making consensus by stubborn nodes. It is found out that \textit{RR} network makes it slow to have consensus by stubborn nodes. 
  
\section{Discussion} 
So far, the competitions of two-layers network has been researched and analyzed under various conditions. It was found out that how network structures have the influence on the consensus of two-layers, how the updating rules affect the state of network, what nodes have more influential to affect the network state, and which method is more effective way to identify important nodes. Through these results, the state of interconnected networks may be controlled by managing the number of degrees and the method of updating rules. And as the best and fastest way to change the state of networks, the important nodes could be recognized and controlled by using the method to find key nodes.
In real world, we can find out the phenomenon of these competitions, such as election, legislation, adoption of new policies and making decision on social conflict issues. These competitions of real world may have similar characteristics with our simulation results. Therefore, based on simulation results, these competitions can be applied to solve the social conflict. As future work, it could be very interesting to make generalized competition models with various structures and updating rules and find key nodes on generalized competition models.   
