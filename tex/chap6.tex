%# -*- coding: utf-8-unix -*-
% !TEX program = xelatex
% !TEX root = ../thesis.tex
% !TEX encoding = UTF-8 Unicode
%%==================================================
%% chapter02.tex for SJTU Master Thesis
%% based on CASthesis
%% modified by wei.jianwen@gmail.com
%% Encoding: UTF-8
%%==================================================

\chapter{Conclusion}
\label{chap6}
We have researched the competition of two layer networks. By changing network structures, switching updating rules and selecting key nodes, the features of competition on two layers have been found out. We hope that deficiency of this research would be researched forward and developed. 
\section{Summary}
So far, many simulations have been carried out. In summary, it could be arranged as follows. 
In chapter.\ref{chap2}, interconnected networks with different dynamics on each layer were introduced to understand the competition on interconnected network.  And some indexes were provided to measure how the state of network is changed and to evaluate the consensus on two-layers. Based on this modeling, various simulations have been implemented according to 3 main topics as follows.
\begin{itemize}
\item Competition on two-layers network with various structures
\item Competition on with different updating rules
\item Key nodes selection on two-layers network
\end{itemize}
In chapter.\ref{chap3}, we have investigated competition dynamics on two-layers network with various structures. With changing network structures, it has been measured and evaluated that how the state of network was changed and whether the networks make consensus or not. As the method to revise the network structure, 3 ways were provided such as changing internal degrees, changing external degrees and switching network types. 
First, as the result of changing the internal degrees, it could be found out that internal degrees on each layer has different features. The number of internal degrees on layer A has the tendency to keep a positive state and to change a negative state into a positive state. And the number of internal degrees on layer B has the tendency to hinder a positive consensus state. Second, as the result of changing the external degrees, \textit{Hierarchical Models} were provided. \textit{Hierarchical Models} show that it is easy to make consensus on both layers when the number of external edges in decision making layer is more than opinion layer and the number of nodes in decision making layer is less than opinion layer. Third, as the result of switching the network type, there is no obvious difference on the final state of network. That means if there are no stubborn nodes, network types do not matter. However, it is found out that the number of internal edges has more influential role for changing the state of network than network types.\\

In chapter.\ref{chap4}, it has been researched that how the updating rules have influence on the competition of two-layers network. Though updating rules are very various, we just have considered time-related updating rules, such as simultaneous updating rule and sequential updating rule. According to where the updating rules are applied, we have implemented the simulations of 3 categories, order of layers, order of nodes and order of links. Through simulation results, several conclusions could be arranged. First, dynamics order between layers does not have an significant influence for changing the state of network. Second, order of edges in layer A, that can be analyzed as characteristics of nodes such as rash and considerate, has a vital influence for determining the final state of network such as same orientation consensus, coexistence and opposite orientation consensus. Third, order of nodes in layer B, that can be analyzed as communication method, is more influential for changing the state of network than order of nodes in layer A because it makes fast opinion convergent or slow opinion convergent. That means the communication method in decision making layer is very important for determining consensus time. Fourth, networks with simultaneous updating rules are easy to make slow consensus and coexistence or to change into the opposite state, otherwise networks with sequential updating rules are easy to make fast consensus.\\

In chapter.\ref{chap5}, it has been studied that how the key nodes could be selected on the various two-layers networks. To select key nodes on the various networks, we used single indicators and multiple indicators on various networks described in chapter.\ref{chap3}. Through the simulation results, several conclusions could be arranged as follows. First, the most effective method to identify key nodes is different according to network structures and layers as shown in Table.~\ref{effective methods}. Second, as single indicators, pagerank, degree and betweenness work well for selecting key nodes. Third, as multiple indicators, combined node centrality totally has good results to recognize the key nodes on various interconnected networks. Fourth, the more number of links on layer A makes it easy to have consensus by stubborn nodes, and the more number of links on layer B makes it hard to make consensus by stubborn nodes. Fifth, as shown in \textit{Hierarchical Models}, decreasing the number of nodes on layer B and increasing the number of external edges on layer B make it hard to identify key nodes on layer B and make it easy to have consensus by stubborn nodes. Sixth, network types have the influence on whether the network can make consensus by stubborn nodes or not. Especially, it is found out that \textit{RR} network is harder to make consensus by stubborn nodes and to select key nodes than \textit{BA} network. \\
  
\section{Discussion} 
So far, the competition of two-layers network has been researched and analyzed under various conditions. It has been found out that how network structures have the influence on the consensus of two-layers, how the updating rules affect the state of network, what nodes have more influential to affect the state of network, and which method is more effective way to identify important nodes. Through these results, the state of two-layers network might be controlled by managing the number of edges and the method of updating rules. And for the best and fastest way to change the state of networks, the important nodes might be recognized and controlled by using the method to select key nodes.
In real world, we can find out the phenomenon of these competitions, such as election, legislation, adoption of new policies and making decision on social conflict issues. These competitions of real world may have similar characteristics with our simulation results. Therefore, based on simulation results, these competition models can be applied to solve the social conflicts. As future work, it could be very interesting to make generalized competition models with various structures and updating rules, and to recognize key nodes on generalized competition models.   
