%# -*- coding: utf-8-unix -*-
% !TEX program = xelatex
% !TEX root = ../thesis.tex
% !TEX encoding = UTF-8 Unicode
%%==================================================
%% chapter02.tex for SJTU Master Thesis
%% based on CASthesis
%% modified by wei.jianwen@gmail.com
%% Encoding: UTF-8
%%==================================================

\chapter{Conclusion}
\label{chap:conclusion}
We have researched the competition of two layer networks. By changing network structures, switching time-related updating rules and finding key nodes, the features of competition on two layers have been found out. We hope that deficiency of this research would be researched forward and developed. 
\section{Summary}
So far, many simulations have been carried out. In summary, it could be arranged as follows. 
To begin with, competing interconnected networks with different dynamics on each layer were introduced to understand the dynamics of competition on interconnected network.  And some indexes were provided to measure how the network state is changed and to evaluate the consensus on two layer. Based on this modeling, various simulations have been implemented according to 3 main topics as follows.
\begin{itemize}
\item Competition on two layer with networks of various structures
\item Competition on two layer with time-related updating rules
\item Finding key nodes on two-layer networks
\end{itemize}
In chapter 3, we have investigated competitions on two layer with networks of various structures. With changing network structures, it has been measured and evaluated that how the interconnected network changed its state and made consensus. As the method to revise the network structure, 3 ways were provided such as changing internal degrees, changing external degrees, and switching network types. 
First, as the result of changing the internal degrees, it could be found out that internal degrees on each layer has different features. The number of internal degrees on layer A has the tendency to keep positive state and to change negative state into positive state. And the number of internal degrees on layer B has the tendency to hinder positive state. Second, as the result of changing the external degrees, Hierarchical Models were provided. Hierarchical Models show that it is easy to make consensus on both layers when the number of external edges in decision making is more than opinion layer. Third, as the result of switching the network type, there is no obvious result of network state change. If there are no stubborn nodes, network types do not matter. The number of internal degrees have more influential role for changing the state than network types.\\

In chapter 4, it has been researched that how the time-related updating rules have influence on the competition of two-layers network. Though updating rules are very various, we just have considered time-related updating rules, simultaneous rule or sequential rule. According to where the time-related updating rules are applied, we have implemented the simulations of 3 categories, order of layers, order of nodes and order of links. Through simulation results, several conclusions could be arranged. First, dynamics order between layers does not have an influence for network state. Second, order of edges in layer A, that can be explained as characteristics of nodes such as rash and considerate, has vital influence for determining the final state of network such as same orientation consensus, coexistence and opposite orientation consensus. Third, order of nodes, that can be analyzed as communication method is more influential for network states than order of nodes in layer A because it makes fast or slow convergent. That means the communication method in decision making layer is very important for determining consensus time. Fourth, networks with simultaneous updating rules are easy to make slow consensus or coexistence and to change into opposite state, otherwise networks with sequential updating rules are easy to make fast consensus. \\

In chapter 5, it has been studied that how the key nodes can be found out on the various interconnected networks. To find key nodes on the network, we use single indicators and multiple indicators on various networks described in chapter.\ref{chap:competition on two layer with networks of various structures}. Through simulation results, several conclusions could be arranged as follows.
First, the most effective method to identify key nodes is different according to network structures and layers as shown in Table.~\ref{effective methods}. Second, as single indicators, pagerank, degree and betweenness work well for finding key nodes. Second, as multiple indicators, combined node centrality totally has good results to recognize the key nodes on various interconnected networks. Third, decreasing the number of links on layer A makes it hard to find key nodes and to have consensus by stubborn nodes.  Fourth, decreasing the number of nodes on layer B makes it hard to identify key nodes and makes it easy to have consensus by stubborn nodes. Fifth, network types have the influence on making consensus by stubborn nodes. Especially, \textit{RR} network makes slower consensus by stubborn nodes than \textit{BA} network and makes it hard to find key nodes. 
  
\section{Discussion} 
So far, the competitions of two-layers network has been researched and analyzed under various conditions. It has been found out that how network structures have the influence on the consensus of two-layers, how the time-related updating rules affect the state of network, what nodes have more influential to affect the network state, and which method is more effective way to identify important nodes. Through these results, the state of interconnected networks might be controlled by managing the number of edges and the method of time-related updating rules. And for the best and fastest way to change the state of networks, the important nodes might be recognized and controlled by using the method to identify key nodes.
In real world, we can find out the phenomenon of these competitions, such as election, legislation, adoption of new policies and making decision on social conflict issues. These competitions of real world may have similar characteristics with our simulation results. Therefore, based on simulation results, these competition models can be applied to solve the social conflicts. As future work, it could be very interesting to make generalized competition models with various structures and updating rules and find key nodes on generalized competition models.   
