%# -*- coding: utf-8-unix -*-
% !TEX program = xelatex
% !TEX root = ../thesis.tex
% !TEX encoding = UTF-8 Unicode

\begin{bigabstract}
	
Different groups usually have different opinions, such as opposite opinions during votes on social issues, presidential campaigns, and so on, where competition is unavoidable. Competition on the interconnected networks has always been a hot topic in the field of complex networks. In this paper, we investigate the competition on a two-layer network and study the influence of network structures, updating rules as well as key nodes on the competition results.

First, the influence of network structures is studied by switching internal degrees, external degrees and network types. A two-layer network is used to model the competition of two groups, where layer A is an opinion formation group, and layer B is a decision-making group. Starting with a polarized competition state, where all nodes in layer A have positive opinions while all nodes in layer B have negative opinions, the state of a network is changed through the evolution of opinions. Various network structures are simulated and analyzed. Simulation results show that both internal and external links play vital roles in the competition. Notably, increasing the number of external and internal links on one layer can make it easy to prevail over the other group and reach consensus.

Second, the influence of updating rules is investigated based on the previous two-layer opinion model. The updating rules, including sequential order and simultaneous order, are considered according to different levels, such as layers, nodes, and edges. It is observed that a simultaneous updating rule is more likely to make the state of a network have a coexistence state and easily be changed to the opposite state, while a sequential updating rule can enable consensus more quickly.

Moreover, the influence of critical nodes on the competition is studied by fixing their states during the evolution of opinions. Several centrality indexes, including Pagerank, degree, eigenvector, betweenness, closeness, and their combinations, are used to select the key nodes. Through simulations, it is found that the influence of key nodes is different according to network structures and opinion dynamics. Besides, both single centrality and multiple centralities have an excellent performance for selecting key nodes, so that the selected critical agents persuade the other group of agents to change their opinion more quickly.

This research can give some insights into several questions, such as how many leaders we need in the organization, how we make the structures of our group, which communication method is better for our group, and who is the most powerful leader in our group. Through various simulation results, the state of a two-layer network could be controlled by managing the number of edges and the method of updating rules. Furthermore, critical nodes might be recognized and controlled by using the method to select key nodes. Finally, those facts give some preliminary results.\\ 

\end{bigabstract}