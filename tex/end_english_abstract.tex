%# -*- coding: utf-8-unix -*-
% !TEX program = xelatex
% !TEX root = ../thesis.tex
% !TEX encoding = UTF-8 Unicode
\begin{bigabstract}
Social conflict can be explained with competition network of two layers. This paper is investigated for a model with the competition between two-layer opinions, where the first layer is opinion formation and the second layer is decision making, on interconnected networks. Networks show the two interacting social sectors, the civilians, and representatives. Layer A is civilian opinion layer consists of four states $(-2, -1, +1, +2)$. These states describe the level of influence of opinion dynamics with reinforcement parameter $\gamma$. The layer B is the decision making layer that consists of only two states $(+1, -1)$.  This layer can influence the decision dynamics with the probability in which decision is proportional to the number of interaction with the opposite opinion population raised to the power of $\beta$. Starting with a polarized competition case, layer A is all positive and layer B is all negative. In this paper, we create new models by changing the network structure, and compare these models with the pre-existing model. Then conditions are investigated that have the influence to opposite side and that make consensus in the interconnected network. This study could help to analyze social networks, such as legalization of social issues and prediction of vote results. Further more, it could contribute to solving the social conflict.
\end{bigabstract}