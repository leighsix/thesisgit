%# -*- coding: utf-8-unix -*-
% !TEX program = xelatex
% !TEX root = ../thesis.tex
% !TEX encoding = UTF-8 Unicode

\begin{bigabstract}
	
Different groups usually have different opinions, such as opposite opinions during votes on social issues, presidential campaigns, and so on, where competition is unavoidable. Competition on the interconnected networks has always been a hot topic in the fields of complex networks and social behavior.  In this paper, we investigate competition in a two-layer network, and study the influence of network structures, updating rules as well as key nodes on the competition results.

First, a two-layer network is used to model the competition of two groups, where layer A is opinion formation, and layer B is decision-making. Starting with a polarized competition state, where nodes in layer A all have positive opinions while nodes in layer B all have negative opinions, the influence of network structures, a number of internal links and external links are analyzed.  Simulation results show that both internal and external links play vital roles in the competition. Notably, increasing the number of external and internal links on one layer can make it easy to prevail over the other group and reach consensus.

Second, the influence of updating rules is investigated based on the previous two-layer opinion model. The updating rules, including sequential order and simultaneous order, are considered according to different levels, such as layers, nodes, and edges. It is observed that a simultaneous updating rule is more likely to have a coexistence state and can be changed to the opposite state, while a sequential updating rule more easily enables fast consensus.

Moreover, the influence of critical nodes on the competition is studied, which are nodes whose states are fixed during the evolution of opinion. Some centrality indexes, including Pagerank, degree, eigenvector, betweenness, closeness, and their combinations, are used to select the key nodes. Through simulations, it is found that the influence of the key nodes is different according to network structures and opinion dynamics. Besides, indexes with single centrality and multiple centralities for selecting key nodes all have an excellent performance on persuading the other group of agents to change their opinion.\\ \\ 

\end{bigabstract}