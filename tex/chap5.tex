%# -*- coding: utf-8-unix -*-
% !TEX program = xelatex
% !TEX root = ../thesis.tex
% !TEX encoding = UTF-8 Unicode
%%==================================================
%% chapter02.tex for SJTU Master Thesis
%% based on CASthesis
%% modified by wei.jianwen@gmail.com
%% Encoding: UTF-8
%%==================================================

\chapter{Influences of key nodes on competition}
\label{chap5}
In this chapter, it is investigated which nodes have the most important influence for keeping or changing their orientation on a two-layer network. There exist many methods to select key nodes, such as Pagerank, degree centrality, eigenvector centrality, betweenness centrality, and closeness centrality. Moreover, in \parencite{mesgari2015, huang2014}, it has been proved that multiple indicators that use the rank difference of several node centralities are useful to identify key nodes and prevent the slow way to identify critical nodes. Based on these methods, such as single-node centrality(single indicators) and combined node centrality(multiple indicators), it is researched which method is the most effective and the most influential for selecting key nodes.  

\section{Method for selecting key nodes}
\label{sec:method for finding key nodes}
As the initial conditions for selecting key nodes, each layer is made of a \textit{BA} network with $512$ nodes, $K=3$, and $1$ external edge. Each simulation takes $100$ steps for opinion evolution, and $100$ simulations are considered for average results. In order to demonstrate the influence of key nodes, the parameters such as $p$ and $v$ are set to be the opposite consensus state to the initial state of a layer for identifying key nodes. And then the stubborn nodes that do not change their states during the opinion evolution are selected by using methods for selecting key nodes, and the ratio of stubborn nodes is increased until a state of the network is changed into the same consensus state with the initial state of the layer for identifying key nodes. Under these conditions, the most powerful method is the fastest method to reach the same consensus state with the initial state of a layer for selecting key nodes. For example, for selecting key nodes on layer A(positive opinion), the parameters are set to be a negative consensus state. Then, as the stubborn nodes on layer A are selected by node centrality or other methods, and the ratio of the stubborn nodes is increased, a state of the network is gradually changed into a positive state. Inversely for selecting key nodes on layer B(negative opinion), the parameters are set to be a positive consensus state. Then, as the stubborn nodes on layer B are selected by the method for recognizing key nodes, and the ratio of stubborn nodes is increased, a state of the network is gradually changed into a negative state. Here, we find the fastest and most powerful method.

As the method to select stubborn nodes, we use two kinds of indicators, single indicators, and multiple indicators. As single indicators, node centralities are applied, such as Pagerank, degree, eigenvector, closeness, and betweenness. As multiple indicators, combined node centralities that consist of several node centralities are applied.
  
First, here is the way to select key nodes by using a single node centrality.
\begin{enumerate}
	\item All nodes are ranked by five node centralities(Pagerank, degree, eigenvector, closeness, betweenness).
	\item The nodes are deactivated from high ranked order until the state of network has a significant difference, i.e., the stubborn nodes are selected according to high ranked order, and the ratio of stubborn nodes is increased. 
	\item The results are compared according to node centralities. When a node centrality makes the state of network reach the fastest to the opposite consensus state with the initial condition or have the most significant change, it is the most powerful method for selecting key nodes.
	\item As the ratio of stubborn nodes is increased, the summation of $AS$, which represents the `Average States' of a network, is calculated on every single indicator. It is recognized that the larger the $AS$ value is on layer A, the more influential that indicator is, inversely the smaller the $AS$ value is on layer B, the more influential that indicator is.
\end{enumerate}

Furthermore, we research the way to recognize critical nodes by using multiple indicators such as combined node centralities(\textit{PR+DE, PR+BE, DE+BE, PR+DE+BE}). Combined node centralities are made up of several selected node centralities. When it is proven that a node centrality is useful for selecting key nodes through the simulations, it is selected as a factor of combined node centrality. Here, $2$ or $3$ node centralities are selected, such as Pagerank, degree, and betweenness. 
The way to recognize key nodes by using combined node centrality follows like this steps. 
\begin{enumerate}
	\item  Each selected node centrality ranks all nodes. All nodes have the ranks as the number of selected node centralities.  
	\item Combined node centrality is calculated by the summation of all ranks which a node has. 
	\item All nodes are ranked again by combined node centrality. The smaller the combined node centrality is, the higher a node is ranked.        
	\item The nodes are deactivated from high ranked order until the state of network has a significant difference, i.e., the stubborn nodes are selected according to high ranked order, and the ratio of stubborn nodes is increased. 
\end{enumerate}

It has been already proven that a single node centrality has good performance to identify key nodes.\parencite{koschutzki2008, francisco2019, bianconi2018}. However, identifying key nodes by multiple indicators is still an open problem because there are lots of ways to set up and optimize the weight of each node centrality.\parencite{huang2014}  Here, we simplify the method by setting the weights as equal and calculate the summation of ranks. Although our multiple indicators need to be researched further, the multiple indicators are evaluated and compared with single indicators. The method for measuring and evaluating key nodes on \textit{BA-BA} network follows as \ref{layerA} and \ref{layerB}.\\
 
\subsection{Key nodes on layer A}
\label{layerA}
\begin{figure}[!htb]
	\centering
	\includegraphics[width=\hsize]{figure/chap5_keynode_A.png}
	\caption{Key nodes on layer A in \textit{BA(3)-BA(3)} network($p=0.2, v=0.4$): (a) Single indicator methods, (b) Multiple indicator methods}
	\label{chap5_keynode_A}
\end{figure}
To select key nodes on layer A, parameters are set to be negative consensus state like $p=0.2, v=0.4$.  As single indicators, five node centralities(Pagerank, degree, eigenvector, closeness, betweenness) are used, and the influence of randomly selected nodes is also compared with five node centralities. As multiple indicators, $2$ or $3$ node centralities are combined, such as Pagerank, degree, and betweenness, which have good performance as single indicators. In combined node centralities, we denote Pagerank, degree, and betweenness as \textit{PR, DE, BE}. Moreover, when they are combined, the methods are denoted as \textit{PR+DE, PR+BE, DE+BE, PR+DE+BE} by using \textit{+}. 

Fig.~\ref{chap5_keynode_A} shows the simulation result for recognizing key nodes on layer A. As a single indicator, Pagerank has the best performance. The next ranks are degree and betweenness. As multiple indicators, \textit{PR+BE} has the most effective result. The next is \textit{PR+DE}. These two methods of multiple indicators work better than Pagerank. Compared with all methods, the best method is \textit{PR+BE}. It can be found out that some multiple indicators are more useful for selecting key nodes than single indicators. \\

\subsection{Key nodes on layer B}
\label{layerB}
\begin{figure}[!htb]
	\centering
	\includegraphics[width=\hsize]{figure/chap5_keynode_B.png}
	\caption{Key nodes on layer B in \textit{BA(3)-BA(3)} network($p=0.3, v=0.5$): (a) Single indicator methods, (b) Multiple indicator methods}
	\label{chap5_keynode_B}
\end{figure}

To select key nodes on layer B, parameters are set to be positive consensus state such as $p=0.3, v=0.5$. Fig.~\ref{chap5_keynode_B} shows the simulation result for identifying key nodes on layer B. As a single indicator, the most effective way to recognize important nodes is Pagerank centrality. The next ranks are degree and betweenness. As multiple indicators, \textit{PR+DE} has the best performance. Pagerank is the most effective method for selecting key nodes on layer B. But, all multiple indicators work better than degree centrality, the second rank in single indicators. It can be found out that combined node centralities also have a good performance for selecting key nodes, though they are not the best. \\

\section{Key nodes on two-layer with different structures}
In this section, we select the key nodes in the networks with various structures that are described in the chapter.\ref{chap3}. Node centralities and combined node centralities are also used as the methods for selecting key nodes. First, \textit{Hierarchical Model} is applied to identify critical nodes. Second, we consider the case that each layer has a different network type, such as \textit{BA-RR} or \textit{RR-BA} networks. Third, the case is considered that each layer has a different number of internal edges. Layer A can have more internal links, or layer B can have more internal links. Both cases are simulated. \\

\subsection{Key nodes in Hierarchical Model}
\begin{figure}[!htb]
	\centering
	\includegraphics[width=\hsize]{figure/chap5_keynode_HM_A.png}
	\caption{Key nodes on layer A in \textit{Hierarchical Model(8)}($p=0.2, v=0.2$):
		(a) Single indicator methods, (b) Multiple indicator methods}
	\label{chap5_keynode_HM_A}
\end{figure}
\begin{figure}[!htb]
	\centering
	\includegraphics[width=\hsize]{figure/chap5_keynode_HM_B.png}
	\caption{Key nodes on layer B in \textit{Hierarchical Model(8)}($p=0.25, v=0.3$):
		(a) Single indicator methods, (b) Multiple indicator methods}
	\label{chap5_keynode_HM_B}
\end{figure}

As described in the chapter.\ref{chap3}, \textit{Hierarchical Model} is the two-layer network that the number of nodes in layer B is reduced at a specific rate, and the external links from nodes in layer B are increased accordingly. Here, each layer consists of a \textit{BA} network with $k=3$. Layer A has $512$ nodes, and layer B has $64$ nodes. We denote this model as \textit{HM(8) with BA(3)}.

Fig.~\ref{chap5_keynode_HM_A} shows the simulation result of key nodes on layer A. Simulation result represents that \textit{PR+DE} is the best method for recognizing key nodes on \textit{HM(8) with BA(3)}. The next ranks are \textit{PR+BE} and Pagerank. The curve of changing the network states shown in Fig.~\ref{chap5_keynode_HM_A} is more straight than Fig.~\ref{chap5_keynode_A}. That means the speed of changing network states(consensus time) is much faster. 

Fig.~\ref{chap5_keynode_HM_B} shows the simulation result of key nodes on layer B. However, the result is different from other simulation results. The best performance method is a random method. That means node centralities do not work on this model. Furthermore, the curve of changing the network states shown in Fig.~\ref{chap5_keynode_HM_B} is also more straight than Fig.~\ref{chap5_keynode_B}, which means the consensus is much easier, and the consensus time is much shorter. It is found out that the \textit{Hierarchical Model} makes it hard to recognize key nodes on layer B and make it easy to reach a consensus of two-layer by key nodes. \\

\subsection{Key nodes on the two-layer network with different network types}
Here, we consider two types of networks, \textit{BA-RR} and \textit{RR-BA}. The number of internal links on each layer is set up as the same or almost the same number to exclude the influence of internal degrees. These models are compared with the \textit{BA-BA} to find out the influence of network types under the same conditions, such as $p$, $v$, and $the ratio of stubborn nodes$.  

\begin{figure}[!htb]
	\centering
	\includegraphics[width=\hsize]{figure/chap5_keynode_BA_RR_A.png}
	\caption{Key nodes on layer A in \textit{BA(3)-RR(6)} network($p=0.2, v=0.4$):
		(a) Single indicator methods, (b) Multiple indicator methods}
	\label{chap5_keynode_BA_RR_A}
\end{figure}
\begin{figure}[!htb]
	\centering
	\includegraphics[width=\hsize]{figure/chap5_keynode_BA_RR_B.png}
	\caption{Key nodes on layer B in \textit{BA(3)-RR(6)} network($p=0.3, v=0.5$):
		(a) Single indicator methods, (b) Multiple indicator methods}
	\label{chap5_keynode_BA_RR_B}
\end{figure}

First, the \textit{BA-RR} network is investigated. Fig.~\ref{chap5_keynode_BA_RR_A} shows the simulation result of key nodes on layer A.\textit{PR+BE} is the most powerful method. The next rank is Pagerank as a single indicator. Compared with the \textit{BA(3)-BA(3)} shown in Fig.~\ref{chap5_keynode_A}, \textit{BA(3)-RR(6)} has smaller \textit{AS} values and a more gentle curve to change the state of the network. 

Fig.~\ref{chap5_keynode_BA_RR_B} shows the simulation result of key nodes on layer B. Betweenness is the best method for identifying key nodes on layer B in the \textit{BA-RR} network. In this model, the degree centrality is not an exact method for the selection of key nodes because the degree of each node is the same in the \textit{RR} network. However, random and degree method is the third and fourth method for recognizing key nodes. That means other methods except for betweenness do not work for identifying key nodes. Compared with the \textit{BA(3)-BA(3)} shown in Fig.~\ref{chap5_keynode_B}, the \textit{BA(3)-RR(6)} has more massive \textit{AS} values and a more gentle curve to change the state of the network. 

\begin{figure}[!htb]
	\centering
	\includegraphics[width=\hsize]{figure/chap5_keynode_RR_BA_A.png}
	\caption{Key nodes on layer A in \textit{RR(6)-BA(3)} network($p=0.2, v=0.4$):
		(a) Single indicator methods, (b) Multiple indicator methods}
	\label{chap5_keynode_RR_BA_A}
\end{figure}
\begin{figure}[!htb]
	\centering
	\includegraphics[width=\hsize]{figure/chap5_keynode_RR_BA_B.png}
	\caption{Key nodes on layer B in \textit{RR(6)-BA(3)} network($p=0.3, v=0.5$):
		(a) Single indicator methods, (b) Multiple indicator methods}
	\label{chap5_keynode_RR_BA_B}
\end{figure}

Next, the \textit{RR-BA} network is considered. Fig.~\ref{chap5_keynode_RR_BA_A} shows the simulation result of key nodes on layer A. The best method is degree centrality. However, in this model, degree centrality is not meant for recognizing key nodes because all nodes in layer A have the same degree. Here, the reason why degree centrality has excellent performance is analyzed as those dynamics are very efficient because nodes are sequentially changed into the stubborn node and interacted(when nodes have the same node centrality, nodes are changed into stubborn nodes sequentially according to interaction order under given algorithm). Moreover, other single indicators have similar \textit{AS} values with the random method. That means node centralities do not work for identifying key nodes though betweenness has better performance than other methods. Compared with the \textit{BA(3)-BA(3)} shown in Fig.~\ref{chap5_keynode_A}, \textit{RR(6)-BA(3)} has smaller \textit{AS} values and does not reach the opposite consensus yet. 

Fig.~\ref{chap5_keynode_RR_BA_B} shows the simulation result of key nodes on layer B. Pagerank has the best performance. The next rank is \textit{PR+DE}.  Compared with the \textit{BA(3)-BA(3)} shown in Fig.~\ref{chap5_keynode_B} , the \textit{RR(6)-BA(3)} has more massive \textit{AS} values and a more gentle curve to change the state of the network. 

Compared with the \textit{BA-BA} network, both \textit{BA-RR} and \textit{RR-BA} have a more gentle curve line to change the state of the network. It can be analyzed that the \textit{RR} network makes it slow for crucial nodes to change the state of the network and makes it hard to select critical nodes though betweenness has excellent performance on the \textit{RR} network.\\  

\subsection{Key nodes on the two-layer network with different number of internal links}
Next, the case is considered that each layer has a different number of internal edges. In case that layer A has a more massive number of internal links, layer A consists of a \textit{BA} network with $k=4$, but layer B consists of a \textit{BA} network with $k=2$. Inversely, in case that layer B has a more massive number of internal links, layer B consists of a \textit{BA} network with $k=4$, but layer A consists of a \textit{BA} network with $k=2$. 

\begin{figure}[!htb]
	\centering
	\includegraphics[width=\hsize]{figure/chap5_keynode_internal_A.png}
	\caption{Key nodes on layer A in \textit{BA(4)-BA(2)} network($p=0.15, v=0.3$):
		(a) Single indicator methods, (b) Multiple indicator methods}
	\label{chap5_keynode_internal_A}
\end{figure}
\begin{figure}[!htb]
	\centering
	\includegraphics[width=\hsize]{figure/chap5_keynode_internal_B.png}
	\caption{Key nodes on layer B in \textit{BA(4)-BA(2)} network($p=0.2, v=0.4$):
		(a) Single indicator methods, (b) Multiple indicator methods}
	\label{chap5_keynode_internal_B}
\end{figure}

First, the case of more internal links on layer A than layer B is investigated. Fig.~\ref{chap5_keynode_internal_A} shows the simulation result of key nodes on layer A in the \textit{BA(4)-BA(2)} network. Betweenness has the best performance for selecting key nodes. The next ranks are \textit{DE+BE}, \textit{PR+BE}, and \textit{PR+DE+BE}. Compared with the \textit{BA(2)-BA(4)} network shown in Fig.~\ref{chap5_keynode_internal_A2}, the curve of changing the state that is shown in Fig.~\ref{chap5_keynode_internal_A} is much more straight-line. That means consensus time is short, and it is easy to have consensus.

Fig.~\ref{chap5_keynode_internal_B} shows the simulation result of key nodes on layer B in the \textit{BA(4)-BA(2)} network. \textit{PR+DE} is the most powerful method. The next ranks are Pagerank, \textit{PR+DE+BE}, and \textit{PR+BE}. Compared with the \textit{BA(2)-BA(4)} network shown in Fig.~\ref{chap5_keynode_internal_B2}, the curve of changing the state that is shown in Fig.~\ref{chap5_keynode_internal_B} is also more straight-line. 

Compared with the \textit{BA(2)-BA(4)} network, it can be analyzed that more internal edges on layer A make it easy to have a consensus by key nodes. 

\begin{figure}[!htb]
	\centering
	\includegraphics[width=\hsize]{figure/chap5_keynode_internal_A2.png}
	\caption{Key nodes on layer A in \textit{BA(2)-BA(4)} network($p=0.57, v=0.37$):
		(a) Single indicator methods, (b) Multiple indicator methods}
	\label{chap5_keynode_internal_A2}
\end{figure}
\begin{figure}[!htb]
	\centering
	\includegraphics[width=\hsize]{figure/chap5_keynode_internal_B2.png}
	\caption{Key nodes on layer B in \textit{BA(2)-BA(4)} network($p=0.6, v=0.4$):
		(a) Single indicator methods, (b) Multiple indicator methods}
	\label{chap5_keynode_internal_B2}
\end{figure}

Next, the case of more internal links on layer B than layer A is researched. Fig.~\ref{chap5_keynode_internal_A2} shows the simulation result of key nodes on layer A in the \textit{BA(2)-BA(4)} network. However, the simulation results are different from other results because the random method has the best performance. That means node centralities do not work on this model. Compared with the \textit{BA(4)-BA(2)} network shown in Fig.~\ref{chap5_keynode_internal_A}, the curve of changing the state that is shown in Fig.~\ref{chap5_keynode_internal_A2}  is much slower and more gentle.

Fig.~\ref{chap5_keynode_internal_B2} shows the simulation result of key nodes on layer B in the \textit{BA(2)-BA(4)} network. \textit{PR+DE} has the most effective performance. The next ranks are Pagerank, \textit{DE+BE}, and betweenness. Compared with the \textit{BA(4)-BA(2)} network shown in Fig.~\ref{chap5_keynode_internal_B}, the curve of changing the state that is shown in Fig.~\ref{chap5_keynode_internal_B2} is much faster at the beginning but much slower at the end. Besides, consensus does not happen in this model.

Compared with the \textit{BA(4)-BA(2)} network, it can be analyzed that the larger number of internal edges on layer B makes consensus by key nodes hard. Moreover, decreasing internal edges on layer A makes it hard to select key nodes on layer A.\\   

\section{Conclusion}
By using node centrality and combined node centrality, key nodes on each layer have been recognized on networks with various structures. Table.~\ref{effective methods} shows total simulation results for selecting key nodes on various interconnected networks.
 
\begin{table}[!htb]
	\scriptsize
	\centering
	\caption{Effective method for selecting key nodes on various networks}
	\label{effective methods}
	\begin{center}
		\begin{tabular}{c|c|c|c|c|c|c|c|c|c} \hline\hline
		  Div                              & A nodes & B nodes & A edges & B edges & layer & 1st method & 2nd method  & 3rd method  & remarks    \\ \hline \hline
         \multirow{1}{*}{BA(3)-BA(3)}      & 512 	 & 512     & 1,527   & 1,527   & A     & PR+BE      & PR+DE       & Pagerank    &            \\ 
			                               &  	     &         &         &         & B     & Pagerank   & PR+DE       & PR+BE       &		     \\ \hline   
	     \multirow{1}{*}{BA(3)-RR(6)}      & 512     & 512     & 1,527   & 1,536   & A     & PR+BE      & Pagerank    & PR+DE+BE    &            \\
	                                       &         &         &         &         & B     & betweenness& DE+BE       & random      & not working\\ \hline
	     \multirow{1}{*}{RR(6)-BA(3)}      & 512     & 512     & 1,536   & 1,527   & A     & degree     & DE+BE       & betweenness & not working\\ 
	                                       &         &         &         &         & B     & Pagerank   & PR+DE       & PR+BE       &            \\ \hline
		 \multirow{1}{*}{BA(4)-BA(2)}      & 512     & 512     & 2,032   & 1,020   & A     & betweenness& DE+BE       & PR+BE       &            \\ 
		                                   &         &         &         &         & B     & PR+DE      & Pagerank    & PR+DE+BE    &            \\ \hline
		 \multirow{1}{*}{BA(2)-BA(4)}      & 512     & 512     & 1,020   & 2,032   & A     & random     & Pagerank    & PR+DE       & not working\\ 
		                                   &         &         &         &         & B     & PR+DE      & Pagerank    & DE+BE       &            \\ \hline
		 \multirow{1}{*}{HM(8) with BA(3)} & 512     & 64      & 1,527   & 183     & A     & PR+DE      & PR+BE       & Pagerank    &            \\ 
		                                   &         &         &         &         & B     & random     & DE+BE       & PR+DE+BE    & not working\\ \hline
			\hline
		\end{tabular}
	\end{center}
\end{table}

Here, we find several facts from these simulation results. First, it can be found out that the best and most powerful method for select key nodes is different according to network structures and layers. Second, as single indicators, Pagerank, degree and betweenness are an excellent method to select key nodes on a two-layer network. Third, as multiple indicators, combined node centralities have an excellent performance to recognize the critical nodes on various networks. Combined node centralities are the first or second effective methods on all simulation models.(except not working methods)  Fourth, as the results are shown in interconnected networks with a different number of internal edges on each layer, the larger number of links on layer A makes it easy to have a consensus by key nodes, and the larger number of links on layer B makes it hard to make a consensus by key nodes. Besides, decreasing internal edges on layer A makes it hard to recognize key nodes on layer A.  Fifth, as the results are shown in the \textit{HM(8) with BA(3)} network, decreasing the number of nodes on layer B and increasing the number of external edges on layer B make it hard to identify key nodes on layer B and makes it easy to reach consensus by key nodes. Sixth, as the results are shown in interconnected networks with different network types, network types influence whether a network can make consensus by key nodes or not. Notably, it is found out that the \textit{RR} network makes it slow to have a consensus by key nodes and makes it hard to recognize critical nodes. \\



