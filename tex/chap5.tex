%# -*- coding: utf-8-unix -*-
% !TEX program = xelatex
% !TEX root = ../thesis.tex
% !TEX encoding = UTF-8 Unicode
%%==================================================
%% chapter02.tex for SJTU Master Thesis
%% based on CASthesis
%% modified by wei.jianwen@gmail.com
%% Encoding: UTF-8
%%==================================================

\chapter{Finding key nodes on two layer networks}
\label{chap:finding key nodes on two layer networks}
In this chapter, it would be investigated that what nodes are important for keep orientation on two-layer networks. There exist many methods to find key nodes, such as pagerank, degree centrality, and eigenvector centrality. Based on these methods, it would be researched that which method is the most effective and the most influential for changing state on two layers.  

Here is the way to find key nodes on two-layer networks by using centrality. All nodes are ranked by node centrality, and the ratio of unchanged nodes are increased according to ranked order, until the average states of network have different states. 
When the ratio of unchanged nodes according to node centrality is the least, that centrality is the most influential property for interconnected network.
As initial condition for finding key nodes, each layer is made of BA network with 2048 nodes and 1 external edge. 

\section{Key nodes on layer A}



\section{Key nodes on layer B}



\section{Key nodes on two layers with different structures}
