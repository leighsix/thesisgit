%# -*- coding: utf-8-unix -*-
% !TEX program = xelatex
% !TEX root = ../thesis.tex
% !TEX encoding = UTF-8 Unicode
%%==================================================
%% chapter02.tex for SJTU Master Thesis
%% based on CASthesis
%% modified by wei.jianwen@gmail.com
%% Encoding: UTF-8
%%==================================================

\chapter{Competition on two layer with various updating rules}
\label{chap:competition on two layer with different updating rules}
Here, we would control dynamics orders between layers and updating rules of nodes states. With changing dynamics orders and updating rules, it would be investigated how the state of network is changed.
In this chapter, each layer consists of \textit{Barabasi-Albert(BA)} network that has $N$ nodes with attaching new nodes each with $K$ edges that are preferentially attached to existing nodes with high degree as introduced in \parencite{barabasi1999}. Each node of one layer is connected with a random node on the other layer. This means each node has only $1$ external un-directed edge. Simulations are preformed on network with $N=2048$, and $K = 3$.

When considering dynamics order on two-layer networks, there are many ways to update the state of nodes. Dynamics order of two layer can be considered whether layer A works first or layer B. And, nodes can be thought about whether the states of nodes are changed simultaneously or sequentially or randomly. Links connected with a node also can be deliberated whether links are activated on a node sequentially or simultaneously or randomly. But, in layer B dynamics, order of edges in one node is always for simultaneous updating rule, because dynamics formula already considers states of all connected neighbor nodes simultaneously. To sum up, as shown in Table.\ref{table1}, 25 updating rules would be considered according to layers, nodes and edges. 
\begin{table}[htp]
	\scriptsize
	\begin{center}
		\begin{tabular}{c|c|c|c|c}
			Order of layers                                  & \multicolumn{2}{|c|}{Layer A}                                      & Layer B                & remarks   \\ \cline{2-4}
			& Order of nodes                 & Order of edges                     & Order of nodes         &  \\ \hline
			\multirow{10}{*}{Layer A $\rightarrow$ Layer B} & \multirow{4}{*}{Sequential}    & \multirow{2}{*}{Sequential}        & Sequential             & O(o, o) $\to$ D(o) \\  \cline{4-5}  
			&                                &                                    & Simultaneous           & O(o, o) $\to$ D(s) \\  \cline{3-5}     
			&                                & \multirow{2}{*}{Simultaneous}      & Sequential             & O(o, s) $\to$ D(o) \\  \cline{4-5} 
			&                                &                                    & Simultaneous           & O(o, s) $\to$ D(s) \\  \cline{2-5} 
			& \multirow{4}{*}{Simultaneous}  & \multirow{2}{*}{Sequential}        & Sequential             & O(s, o) $\to$ D(o) \\  \cline{4-5}
			&                                &                                    & Simultaneous           & O(s, o) $\to$ D(s) \\  \cline{3-5}
			&                                & \multirow{2}{*}{Simultaneous}      & Sequential             & O(s, s) $\to$ D(o) \\  \cline{4-5}
			&                                &                                    & Simultaneous           & O(s, s) $\to$ D(s) \\  \cline{2-5}
			& \multirow{2}{*}{Random}        & \multirow{2}{*}{Random}            & Sequential             & O(r, r) $\to$ D(o) \\  \cline{4-5}
			&                                &                                    & Simultaneous           & O(r, r) $\to$ D(s) \\   \hline
			\multirow{10}{*}{Layer A $\leftarrow$ Layer B}  & \multirow{4}{*}{Sequential}    & \multirow{2}{*}{Sequential}        & Sequential             & O(o, o) $\leftarrow$ D(o) \\  \cline{4-5}  
			&                                &                                    & Simultaneous           & O(o, o) $\leftarrow$ D(s) \\  \cline{3-5}     
			&                                & \multirow{2}{*}{Simultaneous}      & Sequential             & O(o, s) $\leftarrow$ D(o) \\  \cline{4-5} 
			&                                &                                    & Simultaneous           & O(o, s) $\leftarrow$ D(s) \\  \cline{2-5} 
			& \multirow{4}{*}{Simultaneous}  & \multirow{2}{*}{Sequential}        & Sequential             & O(s, o) $\leftarrow$ D(o) \\  \cline{4-5}
			&                                &                                    & Simultaneous           & O(s, o) $\leftarrow$ D(s) \\  \cline{3-5}
			&                                & \multirow{2}{*}{Simultaneous}      & Sequential             & O(s, s) $\leftarrow$ D(o) \\  \cline{4-5}
			&                                &                                    & Simultaneous           & O(s, s) $\leftarrow$ D(s) \\  \cline{2-5}
			& \multirow{2}{*}{Random}        & \multirow{2}{*}{Random}            & Sequential             & O(r, r) $\leftarrow$ D(o) \\  \cline{4-5}
			&                                &                                    & Simultaneous           & O(r, r) $\leftarrow$ D(s) \\   \hline
			\multirow{2}{*}{Layer A $\leftrightarrow$ Layer B}& \multirow{2}{*}{Simultaneous}& Sequential                         & Simultaneous           & O(s, o) $\leftrightarrow$ D(s) \\ \cline{3-5}
			&                                & Simultaneous                       & Simultaneous           & O(s, s) $\leftrightarrow$ D(s) \\ \hline
			\multirow{3}{*}{Layer A $\Leftrightarrow$ Layer B}& \multirow{2}{*}{Sequential}  & Sequential                         & Sequential             & O(o, o) $\Leftrightarrow$ D(o) \\ \cline{3-5}
			&                                & Simultaneous                       & Sequential             & O(o, s) $\Leftrightarrow$ D(o) \\ \cline{2-5}
			& Random                         & Random                             & Random                 & O(r, r) $\Leftrightarrow$ D(r) \\ \hline
			
		\end{tabular}
	\end{center}
	\caption{25 updating rules according to order of layers, nodes, and edges}
	\label{table1}
\end{table}

In table remarks, 'O(o, o) $\to$ D(s)’ means Opinion layer(node : sequential order updating, edges : sequential order updating) $\to$ Decision Making layer(node : simultaneous updating). And 'O(o, o) $\Leftrightarrow$ D(o)’ means that one node in Opinion layer is updated, and then one node in Decision Making layer is updated, this rule is repeated until all nodes are updated.
Dynamics with 25 updating rules are simulated with parameter $p=0.4$ and $v=0.4$. Simulation results are divided by order of layers, nodes and edges. 

\section{Order of layers}
There exist two layers on interconnected network. And each layer have its own dynamics, such as \textit{M-Model} and \textit{AS-Model}. Two dynamics can be operated simultaneously or sequentially. If they act sequentially, dynamics of layer A can act first or dynamics of layer B can work previously. Otherwise, regardless of layers order, nodes of two layers can interact mutually, i.e. one node in layer A are updated and then one node in layer B are updated until all nodes are updated.  
Considering all situations, there are 4 ways in order of two layers, \textit{Layer A $\to$ Layer B, Layer A $\leftarrow$ Layer B, Layer A $\leftrightarrow$ Layer B(simultaneous), Layer A $\Leftrightarrow$ Layer B(interaction regardless of layers)}. 
\begin{figure}[!htb]
	\centering
	\includegraphics[width=\hsize]{chap4_layerorder.png}
	\caption{Simulation results according to orders of layers}
	\label{chap4_layerorder}
\end{figure}
Fig.~\ref{chap4_layerorder} shows 4 simulation results related to orders of layers. As seen in Fig.~\ref{layerorder}, it is shown that there is little difference between orders of layers. Consensus time and result are almost same, though dynamics order is different. Regardless of dynamics directions, when other conditions, such as order of nodes and edges are same, the dynamics results are also very similar. Dynamics order of layers does not have an significant influence on the network state.  

\section{Order of nodes}
In the simulation model, each layer has 2048 nodes, and each node has interaction with other nodes. Now, interaction order of nodes would be considered. One node can be updated sequentially after neighbor nodes are updated. Otherwise, every node can be updated simultaneously. Simulation results would be different according to interaction order of nodes. In addition, random order between nodes is also simulated. In random order, one edge is selected randomly and updated until all edges are considered regardless of orders in nodes or links. Interaction order of nodes have meaning related to time. If networks have short time to change states, networks follow simultaneous updating rule. However, if networks have enough time to update states, networks follow sequential updating rules. In real world, discussion or conversation with enough time means sequential updating rule of nodes, and vote or election means simultaneous updating rule of nodes. 

\begin{figure}[!htb]
	\centering
	\includegraphics[width=\hsize]{figure/chap4_nodeorder.png}
	\caption{Simulation results according to orders of nodes: comparison between order of nodes under same conditions such as order of layers and edges.}
	\label{chap4_nodeorder}
\end{figure}
Fig.~\ref{chap4_nodeorder} shows simulation results. The results are classified to two categories, fast consensus and slow consensus. It is shown that simultaneous interaction between nodes makes slow consensus. Simultaneous order in layer A does not make large difference, but it make consensus slightly slow. Simultaneous interaction between nodes in layer B have more influence on consensus time than in layer A. Random order has similar results with sequential order and does not make different states. For quick social consensus, both opinion layer and decision making layer need sequential updating rule, such as conversation and discussion.      

\section{Order of edges}
Each node has several edges connected with other nodes. Simulation results can be different according to that edges are activated sequentially or simultaneously. If edges of each node work sequentially, a state of node is changed whenever each edge is activated. However, If edges of a node are activated simultaneously, a state of node would be changed considering all connected nodes. In real world, order of edges in one node can be analyzed as characteristics of nodes. If order of edges is sequential, the node would be rash. If order of edges is simultaneous, the node would be considerate. 
\begin{figure}[!htb]
	\centering
	\includegraphics[width=\hsize]{figure/chap4_edgeorder_explanation.png}
	\caption{one node connected with other nodes changes its state with sequential or simultaneous order of edges}
	\label{edgeorder_explanation}
\end{figure}  
For example, considering the case that one node is connected with other nodes as shown in Fig.~\ref{edgeorder_explanation}, we can think how the state of node change. 
If the edges follow sequential updating rule, it is hard to calculate the probabilities, because the states can change according to sequential order of edges. Therefore, we can get next states of nodes by using computer simulation 

If the edges follow simultaneous updating rule, it needs some assumptions: 
\begin{enumerate}
	\item If the number of activated \textit{prob.p} is more than the number of activated \textit{prob.q}, persuasion dynamics would work. 
	\item If the number of activated \textit{prob.p} is same with the number of activated \textit{prob.q}, the state would be unchanged.
	\item If the number of activated \textit{prob.p} is less than the number of activated \textit{prob.q}, compromise dynamics would work.
\end{enumerate}

Through these assumptions, we can calculate probabilities of changing state in layer by considering all cases like these formula.  

\begin{equation}
\begin{array}{l}
K = \{ k \quad|\quad 0, \cdots ,{n^{ - {S_i}}}\}, \quad L = \{l \quad|\quad 0, \cdots ,{n^{{S_i}}}\},
\quad M = \{m \quad|\quad k-l\}, \\
{P_A}({S_i} \mapsto {{S'}_i}) = \begin{cases}
\mbox{unchanged}(k = l):\sum {{p^{{n^{ - {S_i}}}+m}} \cdot {{(1 - p)}^{{n^{{S_i}}}-m}} \cdot {}_{{n^{{S_{^i}}}}}{C_k} \cdot {}_{{n^{ - {S_{^i}}}}}{C_l}} \\
\mbox{persuasion}(k > l):\sum {{p^{{n^{ - {S_i}}}+m}} \cdot {{(1 - p)}^{{n^{{S_i}}}-m}} \cdot {}_{{n^{{S_{^i}}}}}{C_k} \cdot {}_{{n^{ - {S_{^i}}}}}{C_l}} \\
\mbox{compromise}(k < l):\sum {{p^{{n^{ - {S_i}}}+m}} \cdot {{(1 - p)}^{{n^{{S_i}}}-m}} \cdot {}_{{n^{{S_{^i}}}}}{C_k} \cdot {}_{{n^{ - {S_{^i}}}}}{C_l}} 
\end{cases}
\end{array}
\end{equation}

\begin{figure}[!htb]
	\centering
	\includegraphics[width=\hsize]{figure/chap4_edgeorder.png}
	\caption{Simulation results according to orders of edges: comparison between order of edges under same conditions such as order of layers and nodes}
	\label{edgeorder}
\end{figure}

Fig.~\ref{edgeorder} shows the simulation result according to order of edges. The results are categorized to consensus and coexistence(not reaching consensus) due to order of edges. Sequential updating rule of edges makes consensus, i.e. rash nodes make consensus. But simultaneous updating rule of edges makes it hard to reach consensus, i.e. considerate nodes do not make consensus. It can be analyzed that rash node is easy to be extreme and make consensus, but considerate node is very moderate and hard to reach consensu
 
\section{Comparison and Analysis}
It is found out that there are different simulation results according to orders of layers, nodes, and edges. To sum up all updating rules, they can be categorized into 3 parts, positive consensus, coexistence, and negative consensus as shown in Fig.~\ref{ordertotal}.  
\begin{figure}[!htb]
	\centering
	\includegraphics[width=\hsize]{figure/chap4_ordertotal.png}
	\caption{Total results of 25 updating rules with \textit{AS}}
	\label{ordertotal}
\end{figure}
To clearly classify the state of two-layers, the results can be analyzed by using \textit{CI} as shown in Fig.~\ref{ordertotal2}. There are three branch points. In the first branch point, the results are divided according to whether order of nodes in layer B is sequential or simultaneous. In the second and third branch point, the results are divided according to whether order of edges in layer A is sequential or simultaneous. As the results, there are 4 categories such as fast positive consensus, slow positive consensus, coexistence and slow negative consensus. 
\begin{figure}[!htb]
	\centering
	\includegraphics[width=\hsize]{figure/chap4_ordertotal2.png}
	\caption{Total results of 25 updating rules with \textit{CI}}
	\label{ordertotal2}
\end{figure}

\section{Conclusion}
Through these results, several important facts can be arranged. First, networks with more simultaneous updating rules make slow consensus or coexistence, sometimes make transition to opposite orientation. On the other hands, networks with more sequential updating rules make fast consensus. In other words, if opinion layer has more rash nodes, more time to have some conversation and decision making layer has more time to  discuss topics, the network have more probabilities to make consensus for opinion layer. Second, dynamics order between layers does not have an influence for network state, though there exists tiny consensus time gap. Third, order of nodes in layer B has more influence for network states than order of nodes in layer A. order of nodes in layer B makes the first branch point. But order of nodes in layer A does not make any branch point, though there exists tiny consensus time gap. Forth, order of edges in layer A is very influential so that it makes different network states. So to speak, characteristics of nodes in layer A, such as rash and considerate, affects consensus time and sometimes makes transition to coexistence or opposite orientation. 