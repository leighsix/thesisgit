%# -*- coding: utf-8-unix -*-
% !TEX program = xelatex
% !TEX root = ../thesis.tex
% !TEX encoding = UTF-8 Unicode
%%==================================================
%% abstract.tex for SJTU Master Thesis
%%==================================================


\begin{englishabstract}
Social conflict could be modeled on the basis of competition on two-layers network. To understand and analyze social conflict, we would research the features of a competition model by changing conditions such as network structures and updating rules and selecting key nodes on the networks with various structures. The competition model consists of two-layer opinions, where the first layer is opinion formation layer and the second layer is decision making layer. Starting with a polarized competition case where layer A has all the positive opinion and layer B has all the negative opinion, the states of two-layers network are changed by dynamics that each layer has. Competition results are analyzed and compared by indexes that measure the state of network and consensus. Based on this modeling and analysis method, models under various conditions are simulated.  

By changing network structures such as internal links, external links and network types, it is researched how the network structures have influence on the final state of network. These simulation results show that both internal and external links play a vital role for consensus. Especially, increasing the number of external and internal links on one side layer makes it easy to reach consensus.

By changing updating rules such as dynamics order, sequential updating rule and simultaneous updating rule, it is investigated how the updating rules affect the final state of network and what features each updating rule has. These simulations provide that the network with simultaneous updating rules is easy to have slow consensus and coexistence or to change into the opposite state, on the other hands, the network with sequential updating rules is easy to make fast consensus.

By selecting key nodes on two-layers network, it is analyzed that which method is the most effective and best for recognizing key nodes. To identify key nodes on the various networks, single indicators and multiple indicators are applied. As single indicators, node centralities are used such as pagerank, degree, eigenvector, betweenness and closeness. As multiple indicators, $2$ or $3$ node centralities are combined and calculated. As the simulation results, it is found out that the most effective method to select key nodes is different according to network structures and layers. In addition, not only single indicators but multiple indicators also have good performance for recognizing key nodes.

This study could help to analyze social networks, such as legalization of social issues and vote results. Further more, it could contribute to understanding social conflicts and social network structure.\\ \\

\end{englishabstract}

