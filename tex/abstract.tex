%# -*- coding: utf-8-unix -*-
% !TEX program = xelatex
% !TEX root = ../thesis.tex
% !TEX encoding = UTF-8 Unicode
%%==================================================
%% abstract.tex for SJTU Master Thesis
%%==================================================
%\begin{abstract}
	
%不同的群体通常会有不同的意见,比如在社会问题上的投票、总统竞选等等,竞争是不可避免的。互联网络竞争一直是复杂网络和社会行为领域的研究热点。
%本文研究了两层网络上的竞争,研究了网络结构、更新规则以及关键节点对竞争结果的影响。

%首先,采用两层网络对两个群体的竞争进行建模,其中A层为意见形成层,B层为决策层。从A层节点均持肯定意见,B层节点均持否定意见的极化竞争状态出发,
%分析了网络结构、内部链路数和外部链路数的影响。仿真结果表明,内部和外部环节对竞争都起着至关重要的作用。特别是,增加一个层面上的外部和内部联系,
%可以很容易战胜另一个群体,达成共识。

%其次,在前面的两层意见模型的基础上,研究了更新规则的影响。采用顺序更新和同时更新的规则,从不同层次考虑不同的更新顺序,
%包括层顺序、节点顺序和边缘顺序。发现同步更新规则更可能具有共存状态,并被改变为相反状态,而序贯更新规则更容易实现快速一致性。

%最后,研究了在意见演化过程中,关键节点的状态是固定的,关键节点对竞争的影响。利用Pagerank、degree、eigenvector、betweenness、closeness等中心度指标及其组合选择关键节点。
%通过仿真,发现关键节点对网络结构和观点动态的影响是不同的。此外,具有单中心性和多中心性的选择关键节点的索引,在说服其他代理改变观点方面都有很好的性能。\\ \\

%\end{abstract}


\begin{englishabstract}
	
Different groups usually have different opinions, such as opposite opinions during votes on social issues, presidential campaigns, and so on, where competition is unavoidable. Competition on the interconnected networks has always been a hot topic in the field of complex networks and social behavior.  In this paper, we investigate competition in a two-layer network and study the influence of network structures, updating rules as well as key nodes on the competition results.

First, a two-layer network is used to model the competition of two groups, where layer A is an opinion formation, and layer B is a decision-making. Starting with a polarized competition state, where all nodes in layer A have positive opinions while all nodes in layer B have negative opinions, the influences of network structures, internal degrees, and external degrees are analyzed.  Simulation results show that both internal and external links play vital roles in the competition. Notably, increasing the number of external and internal links on one layer can make it easy to prevail over the other group and reach consensus.

Second, the influence of updating rules is investigated based on the previous two-layer opinion model. The updating rules, including sequential order and simultaneous order, are considered according to different levels, such as layers, nodes, and edges. It is observed that a simultaneous updating rule is more likely to have a coexistence state and can be changed to the opposite state, while a sequential updating rule more easily enables fast consensus.

Moreover, the influence of critical nodes on the competition is studied, which are nodes whose states are fixed during the evolution of opinion. Some centrality indexes, including Pagerank, degree, eigenvector, betweenness, closeness, and their combinations, are used to select the key nodes. Through simulations, it is found that the influence of the key nodes is different according to network structures and opinion dynamics. Besides, both indexes with single centrality and multiple centralities for selecting key nodes have an excellent performance on persuading the other group of agents to change their opinion.\\ \\ 

\end{englishabstract}

