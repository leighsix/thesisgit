%# -*- coding: utf-8-unix -*-
% !TEX program = xelatex
% !TEX root = ../thesis.tex
% !TEX encoding = UTF-8 Unicode
%%==================================================
%% abstract.tex for SJTU Master Thesis
%%==================================================


\begin{englishabstract}
	
Different groups usually have different opinions, such like votes on social issues, presidential campaign and so on, where competition is unavoidable. Competition on interconnected network has always been a hot spot in the fields of complex networks and social behavior.  In this paper, we investigate competition on a two-layer network, and study the influence of network structures, updating rules as well as key nodes on the competition results.

First, a two-layer network is used to model the competition of two groups, where layer A is opinion formation and layer B is decision making. Starting with a polarized competition sate that nodes in layer A all have positive opinions while nodes in layer B all have negative opinions, the influence of network structures, numbers of internal links and external links are analyzed.  Simulation results show that both internal and external links play vital roles for the competition. Especially, increasing the number of external and internal links on one layer could make it easy to prevail over the another group and reach consensus.

Second, the influence of updating rules is investigated based on the previous two-layer opinion model. Different updating orders including layer order, node order and edge order are considered from different levels with sequential and simultaneous updating rules. It is found that simultaneous updating rule is more likely to have coexistence state and be changed to the opposite state, while sequential updating rule is much easier to enable fast consensus.

Last but not least, the influence of the key nodes on the competition is studied, where the states of the key nodes are fixed during the opinion evolution. Some centrality indexes including Pagerank, degree, eigenvector, betweenness, closeness and their combinations are used to select the key nodes. Through simulations, it is found that the influence of the key nodes is different according to network structures and opinion dynamics. In addition, indexes with single centrality and multiple centralities for selecting key nodes, all have good performance on persuading the other group of agents to change their opinion.\\ \\ 

\end{englishabstract}

