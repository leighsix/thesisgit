%# -*- coding: utf-8-unix -*-
% !TEX program = xelatex
% !TEX root = ../thesis.tex
% !TEX encoding = UTF-8 Unicode
%%==================================================
%% abstract.tex for SJTU Master Thesis
%%==================================================


\begin{englishabstract}
Social conflict can be explained with competition network of two layers. This research is investigated for a model with the competition between two-layer opinions, where the first layer is opinion formation and the second layer is decision making, on interconnected networks. Starting with a polarized competition case where layer A has all the positive opinion and layer B has all the negative opinion, competition simulations are considered with various network structures and various updating rules. And it would be also researched that which nodes is more influential to affect the state of network. With Monte Carlos simulations, various structural models are compared with average state and consensus ratio, which shows that both internal and external links play a vital role for consensus. Especially, increasing the number of external and internal links on one side layer make it easy to reach consensus. And it is found out that the final state of networks could be different according to updating rules, which is analyzed as communication methods and the characteristic of nodes. Last, it is provided that which way is the best and fastest to recognize important nodes on various network structures. This study could help to analyze social networks, such as legalization of social issues and prediction of vote results. Further more, it could contribute to understanding social conflicts and social network structure.
\end{englishabstract}

