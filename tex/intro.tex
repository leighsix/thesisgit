%# -*- coding: utf-8-unix -*-
% !TEX program = xelatex
% !TEX root = ../thesis.tex
% !TEX encoding = UTF-8 Unicode
%%==================================================
%% chapter01.tex for SJTU Master Thesis
%%==================================================

%\bibliographystyle{sjtu2}%[此处用于每章都生产参考文献]
\chapter{Introduction}
\label{chap:intro}
People have their own opinions, and sometimes they change their opinions in response to others that hold views on given issues. Their opinions are reflected to the leader to make laws and vital decision. These phenomena can be found out  in some cases, such as voting, legislation and adoption of new policies. It is widely recognized that opinion formation and decision making formation have mutual interaction as interconnected networks.\cite{mikko2013, danziger2019, newman2010, boccaletti2014, domenico2013, tomasini2015, namkhanhvu2017}. And sometimes, opinion formation could be opposed to decision making formation. These situations often make social conflict and cause social confusion. To figure out these social conflicts, it is needed to understand and analyze the competition of interconnected networks. So far, physics and computer science have researched these social conflict by modeling and analyzing the complex systems\cite{huberman2004, zuev2012, laguna2004, masuda2015}. The researches include opinion dynamics, voter model, game theory and many more\cite{bianconi2018}. 
Competition of interconnected networks has been researched in many ways. These networks can be applied to the dissemination of computer viruses, messages, opinions, memes, diseases and rumors\cite{hua2014,shenyu2018, zhou2018, alvarez2016,gomez2015,diep2017,rocca2014,velasquez2018}. Opinion dynamics on interconnected networks are investigated with various network models such as \textit{Abrams-Strogatz(AS)} model\cite{abrams2003,vazquez2010} and $M$ model\cite{rocca2014}. So far, main researches have focused on what factors make a consensus or dissent, which have shown that the system can make positive consensus, negative consensus or coexistence under certain range of parameters, such as volatility, reinforcement, and prestige. Also, it is found out that the thresholds make the transition of states and they can explain and analyze the social phenomena in real world such as the legislation, election result, and social network\cite{alvarez2016, amato2017, diep2017}. In \cite{gomez2015}, it is shown that the transition from localized to mixed status occurs through a cascade from poorly connected nodes in the layers to the highly connected ones. In addition, the main features, such as transition and cascade, found in Monte Carlo simulation are exactly characterized by the mean-field theory and magnetization\cite{alvarez2016, diep2017, amato2017, gomez2015}. Based on the previous researches, we would study the main features of competing two-layer networks by changing network structures. 

\section{Related Work}
this work
The model consists of interconnected two layer networks. One layer has the function of social opinion and its own dynamics. The dynamics of the social opinion layer is a kind of opinion dynamics which are also known as M-model\cite{rocca2014}, that includes compromise function and persuasion function. The other layer also has the function of decision-making and its own dynamics. The dynamics of the decision making layer is the language competition dynamics that are also called as Abrams-Strogatz model\cite{abrams2003, vazquez2010}. And, the initial condition of the two layers is assumed to be in opposite states, social opinion layer has all positive states, decision making layer has all negative states.




\section{Motivation}




\section{Thesis Objective}
In this paper, opinion dynamics of a competing two-layer social network are investigated on the basis of the pre-existed research\cite{alvarez2016, gomez2015, diep2017, rocca2014}.
Researching directions have 4 main topics. First, it would be found out what factors make consensus by changing network structures. Second, it would be analyzed whether dynamics orders have an influence on status of two-layer. Third, it would be investigated which method is the best to identify key nodes based on node centralities. Forth, it would be researched which algorithm is the best to find key edges based on edge properties. 
As the result of pre-existed research, interconnected competition of the social network have been researched by finding the threshold or critical point for consensus\cite{alvarez2016, gomez2015, diep2017}. It has been proved that the system can make positive consensus, negative consensus or coexistence parts in interconnected competition of the social network\cite{alvarez2016}. And it is shown that the number of external degree is very important to change the state of layers\cite{gomez2015}. We develop the previous modeling and research to find out the characteristics of interconnected networks. By switching the network structure of each layer, such as changing the number of nodes or the number of edges, we can see how the consensus or coexistence states change and what conditions make the social consensus. This can help to explain social networks phenomena, such as social conflict between social opinion and the congress. Therefore, this research could be used as a tool for analyzing legislation problems, making efficient decision-making system and solving the social conflict. 

The paper is organized as follows. In section 2, the Basic Model is introduced and the dynamics, that is applied to each layer, are described.  In section 3, the simulation results for the basic model and revised models are presented. In section 4 the characteristics of each model are described through the comparison and analysis. Finally, in section 5, the simulation results will be summarized and our findings are concluded.


\subsection{准备工作}
\label{sec:requirements}

要使用这个模板撰写学位论文,需要在\emph{TeX系统}、\emph{TeX技能}上有所准备。

\begin{itemize}[noitemsep,topsep=0pt,parsep=0pt,partopsep=0pt]
	\item {\TeX}系统:所使用的{\TeX}系统要支持 \XeTeX 引擎,且带有ctex 2.x宏包,以2017年或更新版本的\emph{完整}TeXLive、MacTeX发行版为佳。
	\item TeX技能:尽管提供了对模板的必要说明,但这不是一份“ \LaTeX 入门文档”。在使用前请先通读其他入门文档。
	\item 针对Windows用户的额外需求:学位论文模本分别使用git和GNUMake进行版本控制和构建,建议从Cygwin\footnote{\url{http://cygwin.com}}安装这两个工具。
\end{itemize}

\subsection{模板选项}
\label{sec:thesisoption}

sjtuthesis提供了一些常用选项,在thesis.tex在导入sjtuthesis模板类时,可以组合使用。
这些选项包括:

\begin{itemize}[noitemsep,topsep=0pt,parsep=0pt,partopsep=0pt]
	\item 学位类型:bachelor(学位)、master(硕士)、doctor(博士),是必选项。
	\item 中文字体:fandol(Fandol 开源字体)、windows(Windows 系统下的中文字体)、mac(macOS 系统下的华文字体)、ubuntu(Ubuntu 系统下的文泉驿和文鼎字体)、adobe(Adobe 公司的中文字体)、founder(方正公司的中文字体),默认根据操作系统自动配置。
	\item 英文模版:使用english选项启用英文模版。
	\item 盲审选项:使用review选项后,论文作者、学号、导师姓名、致谢、发表论文和参与项目将被隐去。
\end{itemize}

\subsection{编译模板}
\label{sec:process}

模板默认使用GNUMake构建,GNUMake将调用latemk工具自动完成模板多轮编译:


\subsubsection{格式控制文件}
\label{sec:format}

格式控制文件控制着论文的表现形式,包括sjtuthesis.cfg和sjtuthesis.cls。
其中,“cls”控制论文主体格式,“cfg”为配置文件。

\subsubsection{主控文件thesis.tex}
\label{sec:thesistex}

主控文件thesis.tex的作用就是将你分散在多个文件中的内容“整合”成一篇完整的论文。
使用这个模板撰写学位论文时,你的学位论文内容和素材会被“拆散”到各个文件中:
譬如各章正文、各个附录、各章参考文献等等。
在thesis.tex中通过“include”命令将论文的各个部分包含进来,从而形成一篇结构完成的论文。
对模板定制时引入的宏包,建议放在导言区。

\subsubsection{各章源文件tex}
\label{sec:thesisbody}

这一部分是论文的主体,是以“章”为单位划分的,包括:

\begin{itemize}[noitemsep,topsep=0pt,parsep=0pt,partopsep=0pt]
	\item 中英文摘要(abstract.tex)。前言(frontmatter)的其他部分,中英文封面、原创性声明、授权信息在sjtuthesis.cls中定义,不单独分离为tex文件。
不单独弄成文件。
	\item 正文(mainmatter)——学位论文正文的各章内容,源文件是chapter\emph{xxx}.tex。
	\item 附录(app\emph{xx}.tex)、致谢(ack.tex)、攻读学位论文期间发表的学术论文目录(pub.tex)、个人简历(resume.tex)组成正文后的部分(backmatter)。
参考文献列表由bibtex插入,不作为一个单独的文件。
\end{itemize}

\subsubsection{图片文件夹figure}
\label{sec:fig}

figure文件夹放置了需要插入文档中的图片文件(支持PNG/JPG/PDF/EPS格式的图片),可以在按照章节划分子目录。
模板文件中使用\verb|\graphicspath|命令定义了图片存储的顶层目录,在插入图片时,顶层目录名“figure”可省略。

\subsubsection{参考文献数据库bib}
\label{sec:bib}

目前参考文件数据库目录只存放一个参考文件数据库thesis.bib。
关于参考文献引用,可参考第\ref{chap:example}章中的例子。

