%# -*- coding: utf-8-unix -*-
% !TEX program = xelatex
% !TEX root = ../thesis.tex
% !TEX encoding = UTF-8 Unicode
%%==================================================
%% chapter01.tex for SJTU Master Thesis
%%==================================================

%\bibliographystyle{sjtu2}%[此处用于每章都生产参考文献]
\chapter{Introduction}
\label{chap1}
\section{Introduction}
People have their own subjective opinions, and sometimes they change their opinions in response to others' views on those issues. Their opinions are reflected in the leaders when making laws and decisions. These phenomena can be found in various scenarios such as voting, legislation, and the adoption of new policies. It is widely recognized that both opinion formation and decision-making formation have mutual interaction as interconnected networks\parencite{mikko2014, danziger2019, newman2010, boccaletti2014, domenico2013, tomasini2015, namkhanhvu2017}. Sometimes, opinion formation can be opposed to decision-making formation. These situations often give rise to social conflicts and confusion. In order to solve these social conflicts, it is necessary to understand and analyze the competition on interconnected networks. So far, physicists and computer scientists have researched these social conflicts by modeling and analyzing complex systems\parencite{fangwu2004, zuev2012, laguna2004, masuda2014}. The existing researches have applied multiple methods including opinion dynamics\parencite{amato2017, haibo2017, amato2017, quattrociocchi2014}, voter model\parencite{redner2005, casey2009}, game theory\parencite{smyrnakis2019}, and etc\parencite{bianconi2018}. 
Competition of interconnected networks has been applied to varios contexts, such as the propagation of computer viruses\parencite{serazzi2003}, information\parencite{hua2014}, opinions\parencite{alvarez2016, gomez2015,diep2017,rocca2014, velasquez2018}, memes\parencite{massad2013}, infections\parencite{shenyu2018, zhou2018}, and rumors\parencite{liu2018}. Opinion dynamics on interconnected networks have been investigated with various network models such as \textit{Abrams-Strogatz(AS)} model\parencite{abrams2003,vazquez2010} and $M$ model\parencite{rocca2014}.  Based on the previous researches, this thesis studies the main features of competition on two-layer networks by changing network structures and the updating rules, and selecting the key nodes. It is analyzed and concluded that these different conditions cause different opinion evolution results.

\begin{figure}[!htb]
	\centering
	\includegraphics[width=\hsize]{chap1_topic.png}
	\caption{The example of competition on a two-layer network}
	\label{chap1_topic}
\end{figure}

\section{Competition on interconnected networks}
In this research, we focus on the competition on a two-layer network, i.e. an interconnected network. Fig.~\ref{chap1_topic} shows the example of competition on a two-layer network. Compared with a single-layer network, the interconnected network has two dynamics, two set of parameters, and includes internal edges and external edges, as shown in Fig.~\ref{chap1_singlemulti}. Therefore, the interconnected network interaction is more complex than single-layer network interaction.

\begin{figure}[!htb]
	\centering
	\includegraphics[width=\hsize]{chap1_singlemulti.png}
	\caption{Comparison between a single-layer network and a two-layer network}
	\label{chap1_singlemulti}
\end{figure}

In order to put a two-layer network under competition, each layer should consist of different dynamics and parameters. The network dynamics model is based on previous research\parencite{alvarez2016}. The top layer represents social opinion and simulates its dynamics. Some opinion models provide a social mechanism through the compromise process\parencite{naim2003}. Other opinion models apply the persuasive process\parencite{chau2014}. In this research, the social opinion layer is affected by the opinion dynamics known as M-model\parencite{rocca2014}, which includes both compromise function and persuasion function. The bottom layer represents decision-making and has the function of simulating its dynamics. The dynamics of the decision-making layer is the language competition dynamics which is also called as the \textit{Abrams-Strogatz} model\parencite{abrams2003, vazquez2010, patriarca2012}. This model is useful when choosing only one opinion from two opinions. In order to set the competition condition of these two layers, the initial states of the two layers are assumed to be in opposite states, namely the social opinion layer has all positive states, and the decision-making layer has all negative states\parencite{alvarez2016}.

So far, researchers have mainly focused on finding out which factors lead to consensus or dissent(coexistence), which have shown that the system can make positive consensus, negative consensus, or coexistence under a specific range of parameters, such as volatility, reinforcement, strength, and prestige\parencite{alvarez2016}. Moreover, the interconnected competition of the social network has been studied by finding the threshold or critical point for consensus\parencite{alvarez2016, gomez2015, diep2017}. Also, it has been shown that the thresholds mark the transition of states, and they can explain and analyze the social phenomena in the real world, such as legislation, election, and social conflicts\parencite{alvarez2016, gomez2015, amato2017, diep2017}.

In \parencite{gomez2015}, it is shown that the transition from localized status to mixed status occurs through a cascade from poorly connected nodes in the layers to the highly connected ones, and the external degree is critical to changing the state of the network. Besides, the main features, such as transition and cascade, found in Monte Carlo simulation, are precisely characterized by the mean-field theory and magnetization\parencite{alvarez2016, diep2017, amato2017, gomez2015}.

Based on all these previous researches, the competitions of interconnected networks are analyzed from three main aspects, namely network structures, updating rules, and selection of key nodes. Theoretically, the previous models have already been proven correct using the mean-field theory and magnetization. In this thesis, the proposed models will be analyzed by using computer simulations because applied dynamics switch according to the state of nodes. In these models, practical mathematical tools cannot be applied\parencite{nicolas2017, rainer2002}. Therefore, computer simulations will be implemented. Before simulation, backgrounds for the three topics are explained as follows. 

\begin{figure}[!htb]
	\centering
	\includegraphics[width=\hsize]{chap1_network_type.png}
	\caption{Various structures of the network}
	\label{chap1_network_type}
\end{figure}

First, network structures are investigated. Networks, according to their structures, can mainly be divided into regular networks, random networks\parencite{erdos1960}, small-world networks\parencite{watts1998}, scale-free networks\parencite{barabasi2011}, and others. Fig.~\ref{chap1_network_type} shows the structures of various networks. A regular network has a lattice structure, and each node has the same number of links. A random network is made up of edges such that two nodes are connected with probability $p$ in the systems with $K$ nodes. A small-world network is a type of network in which most nodes are not neighbors with each other, but most nodes can reach all other nodes through a small number of links. A small-world network can be constructed from a regular network by eliminating the edges with probability $p$ and connecting two random nodes that are not previously connected. A small-world network has all characteristics of a regular network and random network. A scale-free network is a model in which the distribution of number of edges follows power function. Examples of a scale-free network are the World Wide Web (WWW), the Internet, movie star networks, protein interactions, metabolism, etc. There are several ways to create a scale-free network. Among them, the most typical way is the \textit{Barabasi-Albert} model. The \textit{Barabasi-Albert} model is growing networks in which nodes continue to be added, and connections between nodes have a preferential attachment. The process of creating this model repeats the following two processes: first, add one node with a constant number of edges to the system; second, edges of the added nodes are connected in proportion to the edge number of the pre-existing nodes. In this work, two types of general networks are applied, the random regular(RR) network and the \textit{Barabasi-Albert}(BA) network. 

Second, dynamics orders and updating rules are also studied. For further understanding of the competition on a two-layer network, it is crucial to investigate the interaction between nodes or layers. Methods of interaction between nodes are various. However, according to time, the ways of interactions can be divided into two categories, simultaneous interaction and sequential interaction\parencite{sirbu2017}. In economics and social networks, it has been proven that different results are derived from simultaneous and sequential interaction\parencite{hoffman2011, dietrich2004}. In \parencite{hoffman2011}, it was researched how experimental subjects update induced prior information when receiving two information signals simultaneously or receiving the same signals sequentially. As of the experimental results, the simultaneous method is very different from the sequential method, and under sequential information, the subject’s mean estimates of the two methods(good news preceding bad news or vice versa) are also significantly different from each other. In conclusion, both the sequencing of process and the order of information matters. Moreover, in \parencite{dietrich2004}, the usual random sequential updating rule is displaced by the simultaneous updating rule under the \textit{Sznajd} model. It is found out that this change makes a complete consensus much more difficult. The reason is that some agents with the simultaneous updating rule receive conflicting messages from different neighbor pairs and thus refuse to change their opinion. In this work, both simultaneous and sequential updating rules are applied to layers, nodes, and links.

Third, network centralities are researched to select key nodes on a two-layer network. Network centrality is a index to measure how close each node is to the center of a network, which answers the question, "What characterizes an important node?". The theory of network centrality was first introduced in the field of social network analysis\parencite{freeman1979}. After that, it has expanded to various areas where is related with the concept of the network and has been used to identify which nodes are important in the network. So far, various criteria for assessing network centrality have been presented. Generally, well-known network centralities include degree centrality, betweenness centrality, closeness centrality, eigenvector centrality, and Pagerank\parencite{koschutzki2008, francisco2018, bianconi2018}. Degree centrality is the simplest but the most reliable index. It is defined as the number of interacting neighbor nodes (or edges).  Betweenness centrality is the notion of the shortest path between two nodes on a network. It is explained as the concept to define two different node sets on the network (set $1$, $2$) and quantify how often each node appears on the shortest path for all combinations of nodes in set $1$ and set $2$. Closeness centrality is derived from the idea that the shorter the path that one node reaches all the other nodes is, the more influential the node is. Eigenvector centrality represents the concept that the more a node is connected with critical nodes, the more critical it is. Pagerank measures the convergent value by repeating the process of propagating each node's influence on the other nodes.
So far, many researchers have tried to select critical nodes in a social network\parencite{eom2015, white2003, mesgari2015, hwang1981, huang2014}. Based on node centrality, some algorithms for identifying key nodes have been proposed. In \parencite{mesgari2015, huang2014}, it has been found out that optimally combining multiple measures of nodal importance may provide a robust tool for identifying key nodes of interest, particularly in large graphs. Here, based on previous research, we select the key nodes using single node centrality and combined node centrality.

In this work, for single indicator methods to select key nodes, network centralities will be applied, including Pagerank, degree, eigenvector, betweenness, and closeness. As multiple indicator methods recognize key nodes, several combined node centralities are applied, including \textit{Pagerank+degree, Pagerank+betweenness, degree+betweenness, Pagerank+degree+betweenness} that are based on single indicators. By using these centralities(Pagerank, degree, eigenvector, closeness, betweenness, and combined node centralities), it is investigated which method is the most influential for changing the state of network on various models.\\  

\section{Motivation and organization}

In this work, opinion dynamics of a competing two-layer social network are investigated based on the pre-existed research\parencite{alvarez2016, gomez2015, diep2017, rocca2014}. We develop modeling and analyzations to find out the characteristics of interconnected networks. 

This research has four main directions to investigate the features of the competition model. First, it is shown how to build competition models and how to measure the consensus for analysis. Second, we find out what factors lead to consensus by changing network structures. Third, it is analyzed how dynamics orders and updating rules influence the state of the two-layer network. Fourth, based on network centralities, it is investigated which method is the most effective to identify key nodes. This research proves that these three factors, namely network structures, updating rules, and key nodes, influence the final state of the network.

This research can help to explain social network phenomena, such as social conflicts between two opinions. Therefore, this study can be used as a tool for making an efficient decision-making system, solving the social conflict, and analyzing social network problems such as law-making, legislation, enactment, and voting. Moreover, we can give some advice on how to organize the relation network, how to update the opinion, and how to choose the leaders.

This paper is organized as follows. In chapter~\ref{chap2}, it is introduced how competition model of the two-layer network is made up and how the dynamics of each layer works. Moreover, some indexes are provided to measure and evaluate the simulation results. In chapter~\ref{chap3}, by changing network structures, it is shown how network structure influences the consensus of the two-layer network. In chapter~\ref{chap4}, considering the dynamics orders and updating rules, simulation results are compared and analyzed. In chapter~\ref{chap5}, it is researched which nodes are critical for affecting the state of the network by using single indicators and multiple indicators. Finally, in chapter~\ref{chap6}, all simulation results are summarized, and our findings are concluded. \\


