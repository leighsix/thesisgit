%# -*- coding: utf-8-unix -*-
% !TEX program = xelatex
% !TEX root = ../thesis.tex
% !TEX encoding = UTF-8 Unicode
%%==================================================
%% chapter02.tex for SJTU Master Thesis
%% based on CASthesis
%% modified by wei.jianwen@gmail.com
%% Encoding: UTF-8
%%==================================================


\chapter{Competition on two-layer network with various structures}
\label{chap3}
In this chapter, based on the competition model described in chapter.\ref{chap2}, simulations would be implemented with changing the network structures. As the basic model, the interconnected network with random regular network on each layer would be provided. And then, the structure of interconnected network would be altered by changing the number of internal edges, the number of external edges and network types. Finally, all simulations would be compared and analyzed with the indexes, \textit{AS total}, \textit{PCR}, \textit{NCR} and \textit{CR}.\\

\section{Competition on Random Regular Networks}
\label{competition on Random Regular Networks}
\begin{figure}[!htb]
	\centering
	\includegraphics[width=\hsize]{chap3_RR(5)_RR(5).png}
	\caption{Competition on random regular network}
	\label{chap3_RR(5)_RR(5)}
\end{figure}
\begin{figure}[!htb]
	\centering
	\includegraphics[width=\hsize]{chap3_RR(5)_RR(5)_2d.png}
	\caption{(a) $p$-\textit{AS} chart according to certain $v$ values. (b) $v$-\textit{AS} chart according to certain $p$ values.}
	\label{chap3_RR(5)_RR(5)_2d}
\end{figure}
\begin{figure}[!htb]
	\centering
	\includegraphics[width=\hsize]{chap3_RR(5)_RR(5)_total.png}
	\caption{\textit{AS} according to all $p$s and $v$s}
	\label{chap3_RR(5)_RR(5)_total}
\end{figure}
In this section, simulation results on two-layer network with random regular networks would be provided to comprehend the competition of two layers. Each layer consists of random regular network that has $N$ nodes with $k$ internal edges as introduced in \parencite{kimsangwoo2012, choi2011, bela2001}. Each node of one layer is connected with a random node on the other layer. This means each node has only $1$ external edge. Simulations are performed on the two-layer network with $N=2048$ and $k=5$ on each layer.

The simulation results are shown in Fig.~\ref{chap3_RR(5)_RR(5)_2d} and Fig.~\ref{chap3_RR(5)_RR(5)_total}.  Fig.~\ref{chap3_RR(5)_RR(5)_2d} presents that how the `Average State'(\textit{AS}) are changed according to the other parameter($v$ or $p$), when one parameter($p$ or $v$) is constant. So we can know that how each parameter works on the network. Fig.~\ref{chap3_RR(5)_RR(5)_total} provides total results with all parameters. Through these figures, the characteristics of network would be analyzed.

Fig.~\ref{chap3_RR(5)_RR(5)_2d}(a) shows that when $p > 0.2$, $0.32 < v < 0.47$, it normally tends to have positive consensus. Here, we could found out that if $v$ is lower or larger than certain values, it doesn't make consensus. In Fig.~\ref{chap3_RR(5)_RR(5)_2d}(b), as $v$ increases, it normally change from negative to positive consensus. But, it is found out that when $p$ is very low($p \le 0.11$), it doesn't make positive consensus.  To sum up, when $p$ is large enough, it tends to make positive consensus. But, when $v$ is small enough, it tends to be changed into negative consensus.

Fig.~\ref{chap3_RR(5)_RR(5)_total} shows the states of two layers according to all $p$s and all $v$s. As previously described in chapter.\ref{chap2}, blue areas are for positive consensus, red areas are for negative consensus, and light colored and white area are for coexistence. And indexes for consensus are also measured. Positive consensus value is $0.2475$, and negative consensus value is $0.1575$. Coexistence value is $1 - CR = 0.5950$. By using these values and figures, this model would be compared with networks of various structures in next section. Through these figures, the characteristics of parameters can be arranged as follows. First, large $p$ tends to make positive consensus and small $p$ tends to make negative consensus. Second, small $v$ tends to make negative consensus and large $v$ tends to make coexistence state. \\

\section{Competition on Networks with different number of external links}
\begin{figure}[!htb]
	\centering
	\includegraphics[width=\hsize]{chap3_HM.png}
	\caption{Competition on \textit{Hierarchical Model}}
	\label{chap3_HM}
\end{figure}
In this section, we consider the influence of external links. Based on \textit{RR-RR} model in section~\ref{competition on Random Regular Networks}, we reduce the number of nodes in layer B at a certain rate and increase the external links from nodes in layer B accordingly as shown in Fig.~\ref{chap3_HM}.  We denote \textit{HM(n)} as a \textit{Hierarchical Model} with a level $n$, which means that the number of nodes in layer B is $1/n$ of the number of nodes in layer A, and the number of external links from node in layer B is $n$ in view that the number of external links from node in layer A is $1$. In other words, each node in layer A has one external edge, but each node in layer B has $n$ external edges for \textit{HM(n)}, which means one node in layer B can interact with $n$ nodes in layer A.
\begin{figure}[!htb]
	\centering
	\includegraphics[width=\hsize]{chap3_HMs_AStotal.png}
	\caption{\textit{AS total} on various \textit{Hierarchical Models}}
	\label{chap3_HMs_AStotal}
\end{figure}
To find out the significant influence of external edges, various \textit{HM(n)s} are simulated.  Totally, the simulation results of $8$ \textit{HM(n)}s, \textit{HM(2), HM(4), HM(8), HM(16), HM(32), HM(64), HM(128), HM(256)} are arranged as shown in Fig.~\ref{chap3_HMs_AStotal}.  
Fig.~\ref{chap3_HMs_AStotal} shows that \textit{HM(2)} has the largest area for coexistence part(light colored and white area) and \textit{HM(256)} has the largest area for consensus part(blue and red area). As $n$ in \textit{HM(n)} is increased, coexistence area is decreased and consensus area is increased. Particularly, positive consensus area is significantly increased, negative consensus area is slightly decreased. 
\begin{figure}[!htb]
	\centering
	\includegraphics[width=\hsize]{chap3_HMs_total.png}
	\caption{Histogram for \textit{PCR, NCR, AS total} of \textit{Hierarchical Models(HM(n))}}
	\label{chap3_HMs_total}
\end{figure}
To clearly find out the difference between models, we use the indexes, \textit{PCR, NCR, AS total}. Fig.~\ref{chap3_HMs_total} shows the results to analyze \textit{HM(n)} with indexes. Blue color bar is for \text{PCR}, red color bar is for \text{NCR}, and green color bar is for \text{AS total}. Comparing \textit{HMs} with \textit{Basic model(RR(5)-RR(5))}, \textit{CR} \textit{PCR} and \textit{AS total} are all increased remarkably. \textit{HMs} have larger part of positive consensus than \textit{RR(5)-RR(5)}. And, \textit{HMs} have smaller part of negative consensus than \textit{RR(5)-RR(5)}. 

In summary, all the \textit{Hierarchical Models} have larger consensus ratio than \textit{Random Regular Networks} Model. However, positive consensus ratio is increased, but negative consensus ratio is decreased. It is found out that as the number of B nodes are decreased by larger ratio, the network makes it easier to have positive consensus and harder to have negative consensus. In real world, it would be analyzed that as the number of leaders is smaller, social conflict are decreased and the opinion is convergent to social opinion(layer A). But, sometimes there are some dangers to ignore the leader opinions(layer B), or to cause more social conflict when there are stubborn leaders, that case would be simulated in chapter.\ref{chap5}. \\

\section{Competition on Networks with different number of internal links}
\begin{figure}[!htb]
	\centering
	\includegraphics[width=\hsize]{chap3_changing_internal_edges.png}
	\caption{Competition on interconnected networks with different internal edges}
	\label{chap3_changing_internal_edges}
\end{figure}
Next, the interconnected networks are simulated with various internal degrees in order to define and evaluate the influence of internal degrees. Random regular network would be applied and the internal degrees on each node is switched to various numbers as shown in Fig.~\ref{chap3_changing_internal_edges}. But, there is no change on external degree, which would be fixed to only $1$. Here, \textit{RR(n)-RR(m)} represents layer A has a random regular network with $n$ internal edges per a node and layer B has random regular network with $m$ internal edges per a node.

\begin{figure}[!htb]
	\centering
	\includegraphics[width=\hsize]{chap3_internal_edge_A_total1.png}
	\caption{Simulation results with different internal degrees on layer A}
	\label{chap3_internal_edge_A_total}
\end{figure}

First, the internal degrees on layer A is changed. The internal degree on layer B is fixed to $5,120$, which means each node has $5$ internal degrees on layer B, and the internal degree on layer A is switched into $2,048$, $3,072$, $4,096$, or $5,120$, which means each node has $2$, $3$, $4$, or $5$ internal degrees on layer A. Fig.~\ref{chap3_internal_edge_A_total} shows the simulation results for changing the internal degree on layer A. As shows in Fig.~\ref{chap3_internal_edge_A_total} (a), as the internal degree on layer A is increased, the red part is decreased and the blue part is increased.  

To clearly compare and analyze the results, the results are presented with the indexes, \textit{PCR, NCR, AS total} in Fig.~\ref{chap3_internal_edge_A_total} (b), which shows that as the internal degree on layer A is increased, negative consensus is decreased and positive consensus is increased. As shown in Fig.~\ref{chap3_internal_edge_A_total}, RR(5)-RR(5) has the largest \textit{PCR}, and RR(2)-RR(5) has the largest \text{NCR}. However, all models in Fig.~\ref{chap3_internal_edge_A_total} have almost same \textit{CR}. It can be analyzed that the internal degree on layer A has the tendency to keep positive state and to change negative state into positive state. 
\begin{figure}[!htb]
	\centering
	\includegraphics[width=\hsize]{chap3_internal_edge_B_total1.png}
	\caption{Simulation results with different internal degrees on layer B}
	\label{chap3_internal_edge_B_total}
\end{figure}
Next, the internal degree on layer B is switched. The internal degree on layer A is fixed to  $5,120$, which means each node has $5$ internal degree on layer A, and the internal degree on layer B is switched into $2,048$, $3,072$, $4,096$, or $5,120$, which means each node has $2$, $3$, $4$, or $5$ internal degree on layer B.  

Fig.~\ref{chap3_internal_edge_B_total} shows the results simulated with changing the internal degree on layer B. As shows in Fig.~\ref{chap3_internal_edge_B_total} (a), as the internal degree on layer B is increased, the blue part is decreased, the white and light color part is increased, and the red part is almost same, though the shape of red area is a little changed.  As shown in Fig.~\ref{chap3_internal_edge_B_total} (b), \textit{RR(5)-RR(2)} has the largest \textit{PCR} and \textit{CR}, and \textit{RR(5)-RR(5)} has the smallest \text{PCR} and \textit{CR}. However, all models in Fig.~\ref{chap3_internal_edge_B_total} have almost same \textit{NCR}. It can be analyzed that the internal degree on layer B has the tendency to hinder the positive consensus state and has the inverse relation with \textit{CR}. As the internal degrees on layer B is increased, \textit{PCR} and \textit{CR} are inversely decreased.

Considering two cases where an internal degree of layer A is changed and where an internal degree of layer B is changed, it is recognized that the role of internal degree on layer A is different with internal degree on layer B. The internal degree on layer A has the function to keep the state of layer A, and the internal degree on layer B has the function to restrain the consensus state of layer A and makes coexistence part. 

\begin{figure}[!htb]
	\centering
	\includegraphics[width=\hsize]{chap3_internal_edge_two_total.png}
	\caption{Simulation results with changing internal degrees on both layers}
	\label{chap3_internal_edge_two_total}
\end{figure}

Next, it is simulated that internal degrees are changed on both layer A and layer B, such as \textit{RR(2)-RR(2), RR(3)-RR(3), RR(4)-RR(4), RR(5)-RR(5)} and \textit{RR(6)-RR(6)}. Through these simulations, it would be found out that how total internal degree on both layer A and layer B affects the interconnected network.

Fig.~\ref{chap3_internal_edge_two_total} shows the influence of internal degree on both layers. As the total internal degree is increased, \textit{CR} is inversely decreased and the ratio of positive consensus(blue area) is increased, but the ratio of negative consensus(red area) is decreased. It can be analyzed that a decrease in \textit{CR} is caused by increase in internal degree on layer B, and an increase in ratio of \textit{PCR} is brought out by an increase in internal degree on layer A. But, when the total internal degrees is increased, \textit{PCR, NCR, CR} indexes are decreased. It can be analyzed that too large internal degree on both layers makes it hard for the state of network to reach consensus. 
\begin{figure}[!htb]
	\centering
	\includegraphics[width=\hsize]{chap3_internal_edge_AB_total.png}
	\caption{Total results with different internal degrees on two layers}
	\label{chap3_internal_edge_AB_total}
\end{figure}

In summary, $3$ main simulations have been implemented to find out the influence of internal degree on interconnected network by changing the internal degree on layer A, changing the internal degree on layer B and changing the internal degrees on both layers. The results are arranged as follows. First, it is found out that internal degrees on layer A has the tendency to keep positive state and to change negative state into positive state. Second,  it is shown that the number of internal degrees on layer B has the tendency to hinder the positive consensus state and has the inverse relation with \textit{CR}. Third, too large internal degree makes it hard for the state of network to reach consensus. 

\begin{figure}[!htb]
	\centering
	\includegraphics[width=\hsize]{chap3_internal_edge_summary.png}
	\caption{Categorizing the state of network according to internal degrees on two layers}
	\label{chap3_internal_edge_summary}
\end{figure}

Fig.~\ref{chap3_internal_edge_AB_total} shows the result for all simulations. Through these simulation results, we can analyze that how the state of network is changed according to the internal degrees. Several conclusions can be arranged as shown in Fig.~\ref{chap3_internal_edge_summary}.  First, it is easy to reach negative consensus when the internal degrees on layer A is relatively small(the internal degrees on layer B doesn't matter). Second, it is easy to make positive consensus when the internal degrees on layer A is relatively large and the internal degrees on layer B is relatively small. Third, social conflict can be caused when the internal degrees on both layers are too large. \\ 

\section{Competition on Networks with different network structures}
\begin{figure}[!htb]
	\centering
	\includegraphics[width=\hsize]{chap3_changing_structure_type.png}
	\caption{Competition on networks with different structures}
	\label{chap3_changing_structure_type}
\end{figure}
So far, the interconnected networks have been simulated with only \textit{RR(random regular networks)} that has the same number of edges for each node. Now, the simulations would be implemented on different network types. Here, we use \textit{Barabasi-Albert network(BA)} structure as introduced in \parencite{barabasi1999}. \textit{Barabasi-Albert(BA)} network has $N$ nodes with attaching new nodes each with $K$ edges that are preferentially attached to existing nodes with large degrees. But, there is no change on the external degree, which would be fixed to only $1$.

\begin{figure}[!htb]
	\centering
	\includegraphics[width=\hsize]{chap3_changing_network_type11.png}
	\caption{Simulation results with different network types}
	\label{chap3_changing_network_type1}
\end{figure}

To evaluate the influence of network structure, $4$ simulations are implemented with switching network structures. The \textit{BA} or \textit{RR} network is applied for both layers or switched on each layer. To restrain the influence of internal degree, the number of internal edges in \textit{BA} is set up to be similar with the number of internal edges in \textit{RR}. So, simulations are implemented with $K=6$ on \textit{RR} network and $K=3$ on \textit{BA} network. The number of internal edges in \textit{BA} is $6,135$, and the number of internal edges in \textit{RR} is $6,144$.

\begin{figure}[!htb]
	\centering
	\includegraphics[width=\hsize]{chap3_changing_network_type2.png}
	\caption{Simulation result of BA-BA networks with different internal degrees}
	\label{chap3_changing_network_type2}
\end{figure}

The simulation results are shown in Fig.~\ref{chap3_changing_network_type1}. The results of all simulations have almost the same features. The gap of \textit{PCR, NCR} and \textit{CR} is less than $0.02$. The structure of network make no obvious difference of consensus results. 

Next, the number of internal edges would be increased on the network, where consists of two \textit{BA} networks. It would be found out that how the number of internal edges work on the different network type with \textit{RR} network. $2$ models, \textit{BA(3)-BA(3)} and \textit{BA(5)-BA(5)} would be simulated. \textit{BA(3)-BA(3)} model has $6,135$ internal edges on each layer, and \textit{BA(5)-BA(5)} model has $10,215$ internal edges on each layer.

As shown in Fig.~\ref{chap3_changing_network_type2}, \textit{BA(5)-BA(5)} has larger coexistence area than \textit{BA(3)-BA(3)} because of too many internal edges. It is shown that the influence of internal and external degrees is more important for changing the state of network and making consensus than the influence of network type. However, If there are stubborn nodes on networks, the simulation results would be different because the centrality of stubborn nodes would be changed according to network type. Selecting key nodes on interconnected networks would be simulated and analyzed in chapter.\ref{chap5}.\\

\section{Conclusion}
Various simulations have been simulated to find out the role of internal and external degrees and the influence of network types. All results of simulations are shown in Table.~\ref{Consensus properties of Simulation Models}.
 
\begin{table}[!htb]
	\scriptsize
	\centering
    \caption{Consensus properties of simulation models}
	\label{Consensus properties of Simulation Models}
	\begin{center}
		\begin{tabular}{c|c|c|c|c|c|c|c|c} \hline\hline
			Div                    & A nodes& B nodes & A edges & B edges & AS total  & PCR    & NCR    & CR       \\ \hline \hline
			RR(2)-RR(5)            & 2,048  & 2,048   & 2,048   & 5,120   & -0.3186   & 0.0550 & 0.3175 & 0.3725   \\ \hline
			RR(3)-RR(5)            & 2,048  & 2,048   & 3,072   & 5,120   & -0.1368   & 0.1400 & 0.2925 & 0.4325   \\ \hline
			RR(4)-RR(5)            & 2,048  & 2,048   & 4,096   & 5,120   &  0.0804   & 0.2250 & 0.2275 & 0.4525   \\ \hline
			RR(5)-RR(2)            & 2,048 	& 2,048   & 5,120   & 2,048   &  0.4927   & 0.6725 & 0.1725 & 0.8450   \\ \hline	
			RR(5)-RR(3)            & 2,048 	& 2,048   & 5,120   & 3,072   &  0.3555   & 0.3800 & 0.1525 & 0.5325   \\ \hline
			RR(5)-RR(4)            & 2,048  & 2,048   & 5,120   & 4,096   &  0.2633   & 0.2850 & 0.1525 & 0.4375   \\ \hline
			RR(2)-RR(2)            & 2,048  & 2,048   & 2,048   & 2,048   & -0.1412   & 0.1475 & 0.4050 & 0.5525   \\ \hline
			RR(3)-RR(3)            & 2,048  & 2,048   & 3,072   & 3,072   &  0.0084   & 0.2275 & 0.2825 & 0.5100   \\ \hline
			RR(4)-RR(4)            & 2,048  & 2,048   & 4,096   & 4,096   &  0.1448   & 0.2525 & 0.2075 & 0.4600   \\ \hline
			RR(5)-RR(5)            & 2,048  & 2,048   & 5,120   & 5,120   &  0.2034   & 0.2475 & 0.1575 & 0.4050   \\ \hline
			RR(6)-RR(6)            & 2,048  & 2,048   & 6,144   & 6,144   &  0.2444   & 0.2350 & 0.1375 & 0.3725   \\ \hline
			RR(6)-BA(3)            & 2,048 	& 2,048   & 6,144   & 6,135   &  0.2541   & 0.2275 & 0.1300 & 0.3575   \\ \hline 
			BA(3)-RR(6)            & 2,048 	& 2,048   & 6,135   & 6,144   &  0.2242   & 0.2300 & 0.1425 & 0.3725   \\ \hline
			BA(3)-BA(3)            & 2,048 	& 2,048   & 6,135   & 6,135   &  0.2329   & 0.2200 & 0.1400 & 0.3600   \\ \hline
			BA(5)-BA(5)            & 2,048 	& 2,048   & 10,215  & 10,215  &  0.2496   & 0.1675 & 0.0675 & 0.2350   \\ \hline
			HM(2)  				   & 2,048 	& 1,024   & 5,120   & 2,560   &  0.3073   & 0.3275 & 0.1425 & 0.4700   \\ \hline    
			HM(4) 				   & 2,048 	&  512    & 5,120   & 1,280   &  0.4128   & 0.4125 & 0.1275 & 0.5400   \\ \hline
			HM(8)  				   & 2,048 	&  256    & 5,120   & 640     &  0.4846   & 0.4925 & 0.1150 & 0.6075   \\ \hline
			HM(16)				   & 2,048 	&  128    & 5,120   & 320     &  0.5610   & 0.5800 & 0.1100 & 0.6900   \\ \hline
			HM(32) 				   & 2,048 	&   64    & 5,120   & 160     &  0.5959   & 0.6275 & 0.1025 & 0.7300   \\ \hline
			HM(64) 				   & 2,048 	&   32    & 5,120   & 80      &  0.6185   & 0.6775 & 0.1025 & 0.7800   \\ \hline 
			HM(128) 			   & 2,048 	&   16    & 5,120   & 40      &  0.6379   & 0.7350 & 0.0900 & 0.8250   \\ \hline 
			HM(256) 			   & 2,048 	&    8    & 5,120   & 20      &  0.6454   & 0.7675 & 0.0900 & 0.8575   \\ \hline 
			 \hline
		\end{tabular}
	\end{center}
\end{table} 

Through the simulation results, several facts could be arranged like the followings. If there are no stubborn nodes, network types do not make different result for the state of network and consensus. But, we can provide four conclusions about the roles of internal and external degrees. First, \textit{Hierarchical Models} show that it is easy to make consensus on two-layer when the number of external edges in decision making layer is more than opinion layer and the number of nodes in decision making layer is less than opinion layer. Second, the number of internal edges on layer A has the tendency to keep positive state and to change negative state into positive state. Third, the number of internal edges on layer B has the tendency to hinder the positive consensus state. Fourth, too many internal edges on each layer can cause inner conflict, and that makes it hard to have consensus state. We could apply these facts to make network structures and organizations in real world. \\
